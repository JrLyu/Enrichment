\documentclass[11pt, letterpaper]{article}

\usepackage[utf8]{inputenc}
\usepackage[framemethod=TikZ]{mdframed}
\usepackage[hidelinks]{hyperref}
\usepackage{mathtools, amssymb, amsmath, cleveref, fancyhdr, geometry, , graphicx, float, subfigure, arydshln, url, setspace, framed, pifont, physics, ntheorem}

\geometry{left=2cm, right=2cm, bottom=2cm, top=2cm}

\pagestyle{fancy}
\fancyhead{}
\fancyhead[L]{\leftmark}
\fancyhead[R]{\rightmark}
\fancyfoot{}
\fancyfoot[C]{\thepage}
%\renewcommand{\headrulewidth}{0pt}
\renewcommand{\footrulewidth}{0pt}

\hypersetup{
	colorlinks = true,
	bookmarks = true,
	bookmarksnumbered = true,
	pdfborder = 001,
	linkcolor = blue
}

\newcounter{ndf}[subsection]
\setcounter{ndf}{0}
\newenvironment*{df}[1]{\par\noindent\textbf{Definition \thesubsection.\stepcounter{ndf}\thendf\ (#1).}}{\par}
\newcommand*{\dfref}[1]{Definition \ref{#1}.\thendf}

\newcounter{neg}[subsection]
\setcounter{neg}{0}
\newenvironment*{eg}{\par\noindent\textbf{Example \thesubsection.\stepcounter{neg}\theneg}}{\par}
\newenvironment*{ans}{\par\indent\textbf{\textit{Answer \thesubsection.\theneg}}\par}{\hfill{$\square$}\par}
\newcommand*{\egref}[1]{Example \ref{#1}.\theneg}

\newcounter{nthm}[subsection]
\setcounter{nthm}{0}
\newenvironment*{thm}[1]{\begin{framed}\par\noindent\textbf{Theorem \thesubsection.\stepcounter{nthm}\thenthm\ #1} \par}{\par\end{framed}}
\newcommand*{\thmref}[1]{Theorem \ref{#1}.\thenthm}

\newcounter{nprf}[subsection]
\setcounter{nprf}{0}
\newenvironment*{prf}{\par\noindent\textbf{\textit{Proof \stepcounter{nprf}\thenprf.}}}{\hfill$\blacksquare$\par}
\newenvironment*{dis}{\par\noindent\textbf{\textit{Disproof \stepcounter{nprf}\thenprf.}}}{\hfill$\blacksquare$\par}

\newtheorem*{rmk}{Remark}
\newtheorem*{ext}{Extension}

\newtheorem{cor}{Corollary}[section]
\newtheorem{lem}{Lemma}[section]
\newtheorem{ax}{Axiom}[section]
\newtheorem{nota}{Notation}[section]
\newtheorem{prop}{Proposition}[subsection]
\newtheorem{conj}{Conjecture}[subsection]

\linespread{1.15}

\title{\textbf{Linear Algebra Done Right}}
\author{Jiuru Lyu}
\date{\today}

\def\Z{{\mathbb{Z}}}
\def\R{{\mathbb{R}}}
\def\C{{\mathbb{C}}}
\def\Q{{\mathbb{Q}}}
\def\E{{\mathbb{E}}}
\def\F{{\mathbb{F}}}
\def\d{{\mathrm{d}}}
\def\i{{\mathrm{i}}}
\def\epsilon{\varepsilon}
\def\emptyset{\varemptyset}
\def\st{\emph{ s.t. }}
\def\fs{\emph{ f.s. }}
\def\LI{\mathrm{L.I.}}

\begin{document}
\maketitle
\tableofcontents

\newpage
\section{Vector Spaces}
\subsection{$\R^n$ and $\C^n$}
\begin{df}{Complex Number}
	A \textit{complex number} is an ordered pair $(a,b),$ where $a,b\in\R,$ but we write it as $a+b\i.$
\end{df}
\begin{nota}
	$\C\coloneqq\qty{a+b\i\mid a,b\in\R}$
\end{nota}
\begin{df}{Addition \& Multiplication}
	\[(a+b\i)+(c+d\i)=(a+c)+(b+d)]\i\]
	\[(a+b\i)(c+d\i)=(ac-bd)+(ad+bc)\i\]
\end{df}
\begin{thm}{Properties of Complex Arithmetic}
	\begin{enumerate}
		\item commutativity: $\alpha+\beta=\beta+\alpha;\quad \alpha\beta=\beta\alpha,\quad\forall\alpha,\beta\in\C.$
		\item associativity: $(\alpha+\beta)+\lambda=\alpha+(\beta+\lambda);\quad(\alpha\beta)\lambda=\alpha(\beta\lambda),\quad\forall\alpha,\beta,\lambda\in\C.$
		\item identities: $\lambda+0=\lambda;\quad\lambda\cdot1=\lambda,\forall\lambda\in\C.$
		\item additive inverse: $\forall\alpha\in\C,\exists$ unique $\beta\in\C\st\alpha+\beta=0.$
		\item multiplicative inverse: $\forall\alpha\in\C,\alpha\neq0,\exists$ unique $\beta\in\C\st\alpha\beta=1.$
		\item distributivity: $\lambda(\alpha+\beta)=\lambda\alpha+\lambda\beta,\quad\forall\lambda,\alpha,\beta\in\C.$
	\end{enumerate}
\end{thm}
\begin{df}{Subtraction}
	If $-\alpha$ is the additive inverse of $\alpha,$ \textit{subtraction} on $\C$ is defined by \[\beta-\alpha=\beta+(-\alpha).\]	
\end{df}
\begin{df}{Division}
	For $\alpha\neq0,$ let $\dfrac{1}{\alpha}$ denote the multiplicative inverse of $\alpha.$ Then, \textit{division} on $\C$ is defined by \[\dfrac{\beta}{\alpha}=\beta\cdot\qty(\dfrac{1}{\alpha})\]
\end{df}
\begin{nota}
	$\F$ is either $\R$ or $\C.$	
\end{nota}
\begin{df}{List/Tuple}
	Suppose $n$ is a non-negative integer. A list of length $n$ is an ordered collection of $n$ elements separated by commas and surrounded by parentheses: $(x_1,x_2,x_3,\cdots,x_n).$ Two lists are equal if and only if they have the same length and the same elements in the same order. 
\end{df}
\begin{rmk}
	Lists must have a FINITE length.	
\end{rmk}
\begin{df}{$\F^n$ and Coordinate}
	$\F^n$ is the set of all lists of length $n$ of elements of $\F$: \[\F^n\coloneqq\qty{(x_1,\cdots,x_n)\mid x_i\in\R\forall i=1,\cdots,n},\] where $x_i$ is the $i^\text{th}$ \textit{coordinate} of $(x_1,\cdots,x_n).$
\end{df}
\begin{eg}
	$\R^2=\qty{(x,y)\mid x,y\in\R}$ and $\R^3=\qty{(x,y,z)\mid x,y,z\in\R}.$
\end{eg}
\begin{df}{Addition on $\F^n$}
	\textit{Addition} on $\F^n$ is defined by adding corresponding coordinates: \[(x_1,\cdots,x_n)+(y_1,\cdots,y_n)=(x_1+y_1,\cdots,x_n+y_n).\]	
\end{df}
\begin{thm}{Commutativity of Addition on $\F^n$}
	If $x,y\in\F^n,$ then $x+y=y+x.$
\end{thm}
	\begin{prf}
		Suppose $x=(x_1,\cdots,x_n)$ and $y=(y_1,\cdots,y_n).$ Then \[\begin{aligned}x+y&=(x_1+y_1,\cdots,x_n+y_n)\\&=(y_1+x_1,\cdots,y_n+x_n)=y+x.\end{aligned}\]
	\end{prf}
\begin{df}{Zero}
	Let $0$ denote the list of length $n$ whose coordinates are all $0$: $0\coloneqq(0,\cdots,0).$	
\end{df}
\begin{df}{Additive Inverse on $\F^n$}
	For $x\in\F^n,$ the additive inverse of $x$, denoted $-x,$ is the vector $-x\in\F^n\st x+(-x)=0.$	
\end{df}
\begin{df}{Scalar Multiplication in $\F^n$}
	The product of a number $\lambda\in\F$ and a vector $x\in\F^n$ is computed by multiplying each coordinate of the vector by $\lambda:$ \[\lambda x=\lambda(x_1,\cdots,x_n)=(\lambda x_1,\cdots,\lambda x_n),\] where $x=(x_1,\cdots,x_n)\in\F^n.$
\end{df}
\begin{thm}{Properties of Arithmetic Operations on $\F^n$}
	\begin{enumerate}
		\item $(x+y)+z=x+(y+z)\quad\forall x,y,z\in\F^n$
		\item $(ab)x=a(bx)\quad\forall x\in\F^n$ and $\forall a,b\in\F.$
		\item $1\cdot x=x\quad\forall x\in\F^n$ and $1\in\F.$
		\item $\lambda(x+y)=\lambda x+\lambda y\quad\forall\lambda\in\R$ and $\forall x,y\in\F^n.$
		\item $(a+b)x=ax+bx\quad\forall a,b\in\F$ and $\forall x\in\F^n.$
	\end{enumerate}
\end{thm}

\subsection{Definition of Vector Space}
\begin{df}{Addition on $V$}
	An \textit{addition} on $V$ is a function $(u,v)\mapsto u+v$ for all $u,v\in V.$	
\end{df}
\begin{df}{Scalar Multiplication on $V$}
	A \textit{scalar multiplication} on $V$ is a function $(\lambda,v)\mapsto \lambda v$ for all $\lambda\in\F$ and $v\in V.$	
\end{df}
\begin{df}{Vector Space}
	A \textit{vector space} is a set $V$ along with an addition on $V$ and a scalar multiplication$\st$the following properties hold: 
	\begin{enumerate}
		\item commutativity: $u+v=v+u\quad\forall u,v\in V$
		\item associativity: $(u+v)+w=u+(v+w)$ and $(ab)v=a(bv)\quad\forall u,v,w\in V$ and $\forall a,b\in\F.$
		\item additive identity: $\exists0\in V\st v+0=v\quad\forall v\in V.$
		\item additive inverse: $\exists w\in V\st v+w=0\quad\forall v\in V.$
		\item multiplicative identity: $\exists1\in V\st1\cdot v=v\quad\forall v\in V.$
		\item distributive properties: $a(u+v)=au+av$ and $(a+b)v=av+bv\quad\forall u,v\in V$ and $a,b\in\F.$
	\end{enumerate}
\end{df}
\begin{df}{Vector}
	Elements of a vector space are called \textit{vectors} or points.	
\end{df}
\begin{nota}
	$V$ is a vector space over $\F.$	
\end{nota}
\begin{df}{Real and Complex Vector Space}
	A vector space over $\R$ is called a \textit{real vector space}, and a vector space over $\C$ is called a \textit{complex vector space}.	
\end{df}
\begin{thm}{Unique Additive Identity of Vector Spaces}\label{thm1.2.1}
	A vector space has a unique additive identity. 
\end{thm}
	\begin{prf}
		Suppose $0$ and $0'$ are both additive identities for some vector space $V$. So, \[\begin{aligned}0'&=0'+0\qquad\textit{Since }0\textit{ is an additive identity}\\&=0+0'\qquad\textit{commutativity}\\&=0.\qquad\textit{Since }0'\textit{ is an additive identity}\end{aligned}\]\par Then, $0'=0.$	
	\end{prf}
\begin{thm}{Unique Additive Inverse of Vector Spaces}
	A vector in a vector space has a unique additive inverse.
\end{thm}
	\begin{prf}
		Let $V$ be a vector space. Suppose $w$ and $w'$ are additive inverses of $v$ for some $v\in V$. Note that \[\begin{aligned}w&=w+0\\&=w+(v+w')\\&=(w+v)+w\\&=0+w'=w'.\end{aligned}\]
	\end{prf}
\begin{nota}
	Let $v,w\in V.$ Then, $-v$ denotes the additive inverse of $v$.
\end{nota}
\begin{df}{Subtraction}
	$w-v$ is defined to be $w+(-v).$	
\end{df}
\begin{thm}{}
	$0\cdot v=0\quad\forall v\in V.$
\end{thm}
	\begin{prf}
		Since $v\in V,$ we know \[\begin{aligned}0\cdot v=(0+0)v&=0\cdot v+0\cdot v\\0\cdot v+(-0\cdot v)&=0\cdot+0\cdot+(-0\cdot v)\\0&=0\cdot v\end{aligned}\]	
	\end{prf}
\begin{thm}{}
	$a\cdot0=0\quad\forall a\in\F.$
\end{thm}
	\begin{prf}
		For $a\in\F,$ we have \[\begin{aligned}a\cdot0=a\cdot(0+0)&=a\cdot0+a\cdot0\\a\cdot0+(-a\cdot0)&=a\cdot0+a\cdot0+(-a\cdot0)\\0&=a\cdot0.\end{aligned}\]	
	\end{prf}
\begin{thm}{}
	$(-1)v=-v\quad\forall v\in V.$	
\end{thm}
	\begin{prf}
		For $v\in V,$ we have \[v+(-1)v=1\cdot v+(-1)\cdot v=(1+(-1))\cdot v=0\cdot v=0.\]\par Therefore, by definition, $(-1)v=-v.$	
	\end{prf}
\begin{nota} $\F^S$
	\begin{enumerate}
		\item If $S$ is a set, then $\F^S$ denotes the set of functions from $S$ to $\F$.
		\item For $f,g\in\F^S,$ the \underline{sum} $f+g\in\F^S$ is the function defined by $(f+g)(x)=f(x)+g(x)\quad\forall x\in S.$
		\item For $\lambda\in\F$ and $f\in\F^S,$ the \underline{product} $\lambda f\in\F^S$ is the function defined by $(\lambda f)(x)=\lambda f(x)\quad\forall x\in S.$
	\end{enumerate}
\end{nota}
\begin{thm}{}
	$\F^S$ is a vector space.
\end{thm}


\subsection{Subspace}
\begin{df}{Subspace}
	A subset $U$ of $V$ is called a \textit{subspace} of $V$ if $U$ is also a vector space using the same addition and scalar multiplication as on $V$.
\end{df}
\begin{thm}{Conditions for a Subspace}
	A subset $U$ of $V$ is a subspace of $V$ if and only if $U$ satisfies the following conditions: 
	\begin{enumerate}
		\item additive identity: $0\in U;$
		\item closed under addition: $u,w\in U\implies u+w\in U;$
		\item closed under scalar multiplication: $a\in \F$ and $u\in U\implies au\in U.$
	\end{enumerate}
\end{thm}
	\begin{prf}
		\par ($\Rightarrow$) Suppose $U$ is a subspace of $V$. By definition, $U$ is then a vector space, and so those conditions are automatically satisfied. $\qquad\square$\par 
		($\Leftarrow$)	Suppose $U$ satisfies the three conditions. Since $U$ is a subset of $V$, $U$ automatically has \textit{associativity}, \textit{commutativity}, \textit{multiplicative identity}, and \textit{distributivity}. So, we want to check $U$ has additive inverse and additive identities. \par For additive identity, we know $0\in U,$ by assumption.\par For additive inverse, by condition \#3, we know $-u=(-1)u\in U.$\par Then, $U$ is a vector space. 
	\end{prf}
\begin{eg}
	If $b\in\F$, then $\qty{(x_1,x_2,x_3,x_4)\in\F^4\mid x_3=5x_4+b}$ is a subspace of $\F^4$ if and only if $b=0$.	
\end{eg}
	\begin{prf}
		\par($\Rightarrow$) Suppose $U=\qty{(x_1,x_2,x_3,x_4)\in\F^4\mid x_3=5x_4+b}$ is a subspace of $\F^4.$ Then, $0=(0,0,0,0)\in U.$ So, $0=5\cdot0+b,$ or $b=0.\qquad\square$\par 
		($\Leftarrow$) Suppose $b=0.$ Then, $x_3=5x_4.$ So, $U=\qty{(x_1,x_2,5x_4,x_4)\in\F^4}$
		\begin{enumerate}
			\item[\ding{172}] $0=(0,0,0,0)\in U$
			\item[\ding{173}] Note that \[(x_1,x_2,5x_4,x_4)+(y_1,y_2,5y_4,y_4)=(x_1+y_1,x_2+y_2,5(x_4+y_4),x_4+y_4)\in U\] So, addition is closed under $U$.
			\item[\ding{174}] $\forall a\in\F,$ we have \[a(x_1,x_2,5x_4,x_4)=(ax_1, ax_2, 5(ax_4), ax_4)\in U\] Then, $U$ is a subspace of $\F^4.$
		\end{enumerate}
	\end{prf}
\begin{eg}
	The set of continuous real-valued functions on interval $[0,1]$ is a subspace of $\R^{[0,1]}.$
	\begin{prf}
		\begin{enumerate}
			\item $0$ (zero mapping)$\in U$
			\item Set $f$ and $g\in \mathcal{C}[0,1],$ the set of continuous functions on interval $[0,1].$ Then, $f+g\in\mathcal{C}[0,1].$
			\item From Calculus, we know that $\forall a\in\F,\quad af\in\mathcal{C}[0,1].$
		\end{enumerate}
	\end{prf}	
\end{eg}
\begin{df}{Sum of Subspaces}
	Suppose $U_1,\cdots,U_m$ are subspaces of $V$. The \textit{sum} of $U_1,\cdots,U_m,$ denoted as $U_1+\cdots+U_m,$ is the set of all possible sums of elements of $U_1,\cdots,U_m$: \[U_1+\cdots+U_m=\qty{u_1+\cdots+u_m\mid u_i\in U_i\quad\forall i=1,\cdots,m}.\]	
\end{df}
\begin{eg}
	Suppose $U=\qty{(x,0,0)\in\F^3\mid x\in\F}$ and  $W=\qty{(0,y,0)\in\F^3\mid y\in\F},$ then \[U+W=\qty{(x,y,0)\in\F^3\mid x,y\in\F}.\]
\end{eg}
\begin{thm}{}
	Suppose $U_1,\cdots,U_m$ are subspaces of $V.$ Then, $U_1+\cdots+U_m$ is the \textit{smallest subspace} of $V$ containing $U_1,\cdots,U_m.$	
\end{thm}
	\begin{prf}
		Suppose $U_1,\cdots,U_m$ are subspaces of $U$. Let $U_1+\cdots+U_m=\qty{u_1+\cdots+u_m\mid u_j\in U_j, j=1,\cdots m}.$ Suppose $w_j\in U_j,$ then $w_1+\cdots+w_m\in U_1+\cdots+U_m.$
		\begin{enumerate}
			\item $U_1+\cdots+U_m$ is a subspace of $V$.
			\begin{enumerate}
				\item Note that \[(u_1+\cdots+u_m)+(w_1+\cdots+w_m)=(u_1+w_1)+\cdots+(u_m+w_m)\in U_1+\cdots+U_m,\] so $U_1+\cdots+U_m$ is closed under addition.
				\item Similarly, $U_1+\cdots+U_m$ is closed under scalar multiplication.
				\item Note that $U_j$ is a subspace, so $0\in U_j.$ Hence, $(0,\cdots,0)=0\in U_1+\cdots+U_m$
			\end{enumerate}
			Therefore, we've proven $U_1+\cdots+U_m$ is a subspace of $V.\qquad\square$
			\item Now, we want to show this subspace is the smallest subspace containing $U_1,\cdots,U_m.$ That is, we want to show $\forall\ W\supseteq U_1\cup\cdots\cup U_m,$ we have $W\supseteq U_1+\cdots+U_m$.\par Note that $U_j\subseteq U_1+\cdots+U_m,$ so we have $(U_1\cup U_2\cup\cdots\cup U_m)\subseteq U_1+\cdots+U_m$. This means $U_1+\cdots+U_m$ must contain $U_1,\cdots,U_m.$ Let $W$ be some subspace containing $U_1,\cdots,U_m.$ Then, for $j=1,\cdots,m,$ we have $u_j\in U_j,$ which indicates $u_j\in W.$ Therefore, $u_1+\cdots+u_m\in V$ and thus $U_1+\cdots+U_m\subseteq W.$\par Since $W$ was arbitrary, we've shown $\forall\ W$ that contains $U_1,\cdots,U_m,$ $U_1+\cdots+U_m\subseteq W.$ Therefore, $U_1+\cdots+U_m$ is the smallest. 
		\end{enumerate}
	\end{prf}
\begin{df}{Direct Sum}
	Suppose $U_1,\cdots,U_m$ are subspaces of $V.$ $U_1+\cdots+U_m$ is called a \textit{direct sum} if each element of $U_1+\cdots+U_m$ can be written in only one way as a sum $u_1+\cdots+u_m,$ where $u_j\in U_j.$
\end{df}
\begin{nota}
	If $U_1+\cdots+U_m$ is a direct sum, then we use $U_1\oplus\cdots\oplus U_m$ to denote it.	
\end{nota}
\begin{eg}
	Let $U=\qty{(x,y,0)\in\F^3\mid x,y\in\F}$ and $W=\qty{(0,0,z)\in\F^3\mid z\in\F}.$ Then, $\F^3=U\oplus W.$
	\begin{prf}
		Note that $U+W=\qty{(x,y,z)\mid x,y,z\in\F}=\F^3.$ Suppose \ding{172}: $(x,y,z)=(x,y,0)+(0,0,z),$ for some $x,y,z\in\F$ and \ding{173}: $(x,y,z)=(x',y',0)+(0,0,z')$ for some $x',y',z'\in\F.$ Then, \ding{172}$-$\ding{173}: \[(0,0,0)=(x-x',y-y',0)+(0,0,z-z')=(x-x',y-y',z-z').\] Then, $x-x'=y-y'=z-z'=0,$ which indicates $x=x',\ y=y',\ z=z'.$ So, by definition $U+W$ is a direct sum, or $\F^3=U\oplus W.$
	\end{prf}
\end{eg}
\begin{eg}
	Suppose $U_j$ is the subspace of $\F^n\st$\[\begin{aligned}U_1&=\qty{x,0,0,\cdots,0\mid x\in\F}\\U_2&=\qty{0,x,0,\cdots,0\mid x\in\F}\\&\vdots\\U_n&=\qty{0,0,0,\cdots,x\mid x\in\F}\end{aligned}\] Then, $\F^n=U_1\oplus U_2\oplus\cdots\oplus U_n.$
	\begin{prf}
		
	\end{prf}
\end{eg}
\begin{eg}\label{eg1.3.6}
	Let \[\begin{aligned}U_1&=\qty{(x,y,0)\mid x,y\in\F}\\U_2&=\qty{(0,0,z)\mid z\in\F}\\U_3&=\qty{(0,y,y)\mid y\in\F}\end{aligned}\] Show that $U_1+U_2+U_3$ is not a direct sum.
	\begin{prf}
		
	\end{prf}
\end{eg}
\begin{thm}{}
	Suppose	$U_1,\cdots,U_m$ are subspaces of $V$. Then,$U_1+\cdots+U_m$ is a direct sum if and only if the only way to write $0$ as a sum $u_1+\cdots+u_m$ is by taking each $u_j=0.$
\end{thm}
	\begin{prf}
		\par ($\Rightarrow$)
		($\Leftarrow$)
	\end{prf}
\begin{thm}{}\label{thm1.3.4}
	Suppose	$U$ amd $W$ are subspaces of $V$. Then,$U+W$ is a direct sum if and only if $U\cap W=\qty{0}$.
\end{thm}
	\begin{prf}
		\par ($\Rightarrow$)
		($\Leftarrow$)
	\end{prf}
\begin{rmk}
	When extending \thmref{thm1.3.4} to 3 subspaces $U_1,U+2,U+3,$ we cannot conclude $U_1\oplus U_2\oplus U_3$ if we have $U_1\cap U_2=U_1\cap U_3=U_2\cap U_3=\qty{0}.$ See \egref{eg1.3.6} as a counter example.
\end{rmk}

\newpage
\section{Finite-Dimensional Vector Spaces}
\subsection{Span and Linear Independence}
\subsection{Bases}
\subsection{Dimension}

\end{document}
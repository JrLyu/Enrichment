\documentclass[12pt, a4paper]{article}

\usepackage[utf8]{inputenc}
\usepackage[framemethod=TikZ]{mdframed}
\usepackage[hidelinks]{hyperref}
\usepackage{mathtools, amssymb, amsmath, cleveref, fancyhdr, geometry, , graphicx, float, subfigure, arydshln, url, setspace, framed, pifont, physics, ntheorem, utopia}
%%% for coding %%%
\usepackage{listings}
\usepackage[ruled, vlined, linesnumbered]{algorithm2e}

\geometry{a4paper, left=2cm, right=2cm, bottom=2cm, top=2cm}

\pagestyle{fancy}
\fancyhead{}
\fancyhead[L]{\leftmark}
\fancyhead[R]{\rightmark}
\fancyfoot{}
\fancyfoot[C]{\thepage}
%\renewcommand{\headrulewidth}{0pt}
\renewcommand{\footrulewidth}{0pt}

\hypersetup{
	colorlinks = true,
	bookmarks = true,
	bookmarksnumbered = true,
	pdfborder = 001,
	linkcolor = blue
}


\newcounter{index}[subsection]
\setcounter{index}{0}
\newenvironment*{df}[1]{\par\noindent\textbf{Definition \thesubsection.\stepcounter{index}\theindex\ (#1).}}{\par}

\newenvironment*{eg}{\begin{framed}\par\noindent\textbf{Example \thesubsection.\stepcounter{index}\theindex}}{\par\end{framed}}

\newenvironment*{thm}[1]{\begin{tcolorbox}\par\noindent\textbf{Theorem \thesubsection.\stepcounter{index}\theindex\ #1} \par}{\par\end{tcolorbox}}

\newenvironment*{cor}[1]{\par\noindent\textbf{Corollary \thesubsection.\stepcounter{index}\theindex\ #1}}{\par}
\newenvironment*{lem}[1]{\par\noindent\textbf{Lemma \thesubsection.\stepcounter{index}\theindex\ #1}}{\par}
\newenvironment*{ax}[1]{\par\noindent\textbf{Axiom \thesubsection.\stepcounter{index}\theindex\ #1}}{\par}
\newenvironment*{prop}[1]{\par\noindent\textbf{Proposition \thesubsection.\stepcounter{index}\theindex\ #1}}{\par}
\newenvironment*{conj}[1]{\par\noindent\textbf{Conjecture \thesubsection.\stepcounter{index}\theindex\ #1}}{\par}
\newenvironment*{nota}{\par\noindent\textbf{Notation \thesubsection.\stepcounter{index}\theindex.}}{\par}

\newcounter{nprf}[subsection]
\setcounter{nprf}{0}
\newenvironment*{prf}{\par\indent\textbf{\textit{Proof \stepcounter{nprf}\thenprf.}}}{\hfill$\blacksquare$\par}
\newenvironment*{dis}{\par\indent\textbf{\textit{Disproof \stepcounter{nprf}\thenprf.}}}{\hfill$\blacksquare$\par}
\newenvironment*{sol}{\par\indent\textbf{\textit{Solution \stepcounter{nprf}\thenprf.}}\par}{\hfill{$\square$}\par}

\newtheorem*{hint}{Hint.}
\newtheorem*{rmk}{Remark.}
\newtheorem*{ext}{Extension.}

\linespread{1.25}

\title{Emory University\\\textbf{MATH 212 Differential Equations Learning Notes}}
\author{Jiuru Lyu}
\date{\today}

\def\Z{{\mathbb{Z}}}
\def\R{{\mathbb{R}}}
\def\C{{\mathbb{C}}}
\def\Q{{\mathbb{Q}}}
\def\E{{\mathbb{E}}}
\def\d{{\mathrm{d}}}
\def\i{{\mathrm{i}}}
\def\Arg{{\mathrm{Arg}}}
\def\cis{\mathrm{cis}}
\def\epsilon{\varepsilon}
\def\emptyset{\varemptyset}

\begin{document}
\maketitle

\tableofcontents

\newpage
\section{First Order Ordinary Differential Equations}
\subsection{Introduction}
\begin{df}{Ordinary Differential Equations/ODEs}
	An \textit{ordinary differential equation} is an equation that contains one or more derivatives of an unknown function $y=y(x)$.
\end{df}
\begin{df}{Order of ODEs}
	The \textit{order} of an ODE is the maximum order of the derivatives appearing in the equation.
\end{df}
\begin{df}{Solution to ODEs}
	The \textit{solution} to an ODE is a function $y$ that satisfies the equation.	
\end{df}
\begin{eg}
	Solve $y''=3x+1$.
	\begin{sol}
		\[\begin{aligned}y'&=\int3x+1\ \d x=\dfrac{3}{2}x^2+x+C\\y&=\int y'\ \d x=\int\dfrac{3}{2}x^2+x+C\ \d x=\dfrac{1}{2}x^3+\dfrac{1}{x}x^2+Cx+D.\end{aligned}\]
	\end{sol}
\end{eg}
\begin{df}{Linear ODEs/Non-Linear ODEs}
	A first order ODE is \textit{linear} if it can be written as \[y'+p(x)y=f(x).\] Otherwise, it is \textit{non-linear}.
\end{df}
\begin{df}{Homogenous/Non-Homogenous Linear ODEs}
	If $f(x)=0$, then the linear ODE is \textit{homogenous}. That is, \[y'+p(x)y=0.\] Otherwise, it is \textit{non-homogenous}.
\end{df}
\begin{df}{Trivial/Non-Trivial Solution}
	$y=0$ is a \textit{trivial solution} to a homogenous ODE. Any other solutions are \textit{non-trivial}.	
\end{df}
\begin{df}{One-Parameter Family of Solutions}
	We call $C$ a \textit{parameter} and the equation, therefore solution, defines a \textit{one-parameter family} of solutions.	
\end{df}
\begin{eg}
	For the ODE $y'=1$, $y_1=x+C_1$ is a solution to it, and it is a one-parameter family of solutions. Similarly, for $y'=\dfrac{1}{x^2}$, the one-parameter families of solutions are defined by $y_2=-\dfrac{1}{x}+C_2$ on the interval $(-\infty,0)\cup(0,\infty)$.
\end{eg}



\section{Second Order ODEs}

\section{System of ODEs}
\end{document}
\documentclass[12pt, a4paper]{article}

\usepackage[utf8]{inputenc}
\usepackage[framemethod=TikZ]{mdframed}
\usepackage[hidelinks]{hyperref}
\usepackage{mathtools, amssymb, amsmath, cleveref, fancyhdr, geometry, tcolorbox, graphicx, float, subfigure, arydshln, url, setspace, framed, pifont, physics, ntheorem, utopia}
%%% for coding %%%
\usepackage{listings}
\usepackage[ruled, vlined, linesnumbered]{algorithm2e}

\geometry{a4paper, left=2cm, right=2cm, bottom=2cm, top=2cm}

\pagestyle{fancy}
\fancyhead{}
\fancyhead[L]{\leftmark}
\fancyhead[R]{\rightmark}
\fancyfoot{}
\fancyfoot[C]{\thepage}
%\renewcommand{\headrulewidth}{0pt}
\renewcommand{\footrulewidth}{0pt}

\hypersetup{
	colorlinks = true,
	bookmarks = true,
	bookmarksnumbered = true,
	pdfborder = 001,
	linkcolor = blue
}


\newcounter{index}[subsection]
\setcounter{index}{0}
\newenvironment*{df}[1]{\par\noindent\textbf{Definition \thesubsection.\stepcounter{index}\theindex\ (#1).}}{\par}

\newenvironment*{eg}{\begin{framed}\par\noindent\textbf{Example \thesubsection.\stepcounter{index}\theindex}}{\par\end{framed}}

\newenvironment*{thm}[1]{\begin{tcolorbox}\par\noindent\textbf{Theorem \thesubsection.\stepcounter{index}\theindex\ #1} \par}{\par\end{tcolorbox}}

\newenvironment*{cor}[1]{\par\noindent\textbf{Corollary \thesubsection.\stepcounter{index}\theindex\ #1}}{\par}
\newenvironment*{lem}[1]{\par\noindent\textbf{Lemma \thesubsection.\stepcounter{index}\theindex\ #1}}{\par}
\newenvironment*{ax}[1]{\par\noindent\textbf{Axiom \thesubsection.\stepcounter{index}\theindex\ #1}}{\par}
\newenvironment*{prop}[1]{\par\noindent\textbf{Proposition \thesubsection.\stepcounter{index}\theindex\ #1}}{\par}
\newenvironment*{conj}[1]{\par\noindent\textbf{Conjecture \thesubsection.\stepcounter{index}\theindex\ #1}}{\par}
\newenvironment*{nota}{\par\noindent\textbf{Notation \thesubsection.\stepcounter{index}\theindex.}}{\par}

\newcounter{nprf}[subsection]
\setcounter{nprf}{0}
\newenvironment*{prf}{\par\indent\textbf{\textit{Proof \stepcounter{nprf}\thenprf.}}}{\hfill$\blacksquare$\par}
\newenvironment*{dis}{\par\indent\textbf{\textit{Disproof \stepcounter{nprf}\thenprf.}}}{\hfill$\blacksquare$\par}
\newenvironment*{sol}{\par\indent\textbf{\textit{Solution \stepcounter{nprf}\thenprf.}}\par}{\hfill{$\square$}\par}

\newtheorem*{hint}{Hint.}
\newtheorem*{rmk}{Remark.}
\newtheorem*{ext}{Extension.}

\linespread{1.25}

\title{Emory University\\\textbf{MATH 212 Differential Equations Learning Notes}}
\author{Jiuru Lyu}
\date{\today}

\def\Z{{\mathbb{Z}}}
\def\R{{\mathbb{R}}}
\def\C{{\mathbb{C}}}
\def\Q{{\mathbb{Q}}}
\def\E{{\mathbb{E}}}
\def\d{{\mathrm{d}}}
\def\i{{\mathrm{i}}}
\def\Arg{{\mathrm{Arg}}}
\def\cis{\mathrm{cis}}
\def\epsilon{\varepsilon}
\def\emptyset{\varemptyset}
\def\dsst{\displaystyle}
\def\pqde{\quad\square}

\begin{document}
\maketitle

\tableofcontents

\newpage
\section{First Order ODEs}
\subsection{Introduction}
\begin{df}{Ordinary Differential Equations/ODEs}
	An \textit{ordinary differential equation} is an equation that contains one or more derivatives of an unknown function $y=y(x)$.
\end{df}
\begin{df}{Order of ODEs}
	The \textit{order} of an ODE is the maximum order of the derivatives appearing in the equation.
\end{df}
\begin{df}{Solution to ODEs}
	The \textit{solution} to an ODE is a function $y$ that satisfies the equation.	
\end{df}
\begin{eg}
	Solve $y''=3x+1$.
	\begin{sol}
		\[\begin{aligned}y'&=\int3x+1\ \d x=\dfrac{3}{2}x^2+x+C\\y&=\int y'\ \d x=\int\dfrac{3}{2}x^2+x+C\ \d x=\dfrac{1}{2}x^3+\dfrac{1}{x}x^2+Cx+D.\end{aligned}\]
	\end{sol}
\end{eg}
\begin{df}{Linear ODEs/Non-Linear ODEs}
	A first order ODE is \textit{linear} if it can be written as \[y'+p(x)y=f(x).\] Otherwise, it is \textit{non-linear}.
\end{df}
\begin{df}{Homogenous/Non-Homogenous Linear ODEs}
	If $f(x)=0$, then the linear ODE is \textit{homogenous}. That is, \[y'+p(x)y=0.\] Otherwise, it is \textit{non-homogenous}.
\end{df}
\begin{df}{Trivial/Non-Trivial Solution}
	$y=0$ is a \textit{trivial solution} to a homogenous ODE. Any other solutions are \textit{non-trivial}.	
\end{df}
\begin{df}{One-Parameter Family of Solutions}
	We call $C$ a \textit{parameter} and the equation, therefore solution, defines a \textit{one-parameter family} of solutions.	
\end{df}
\begin{eg}
	For the ODE $y'=1$, $y_1=x+C_1$ is a solution to it, and it is a one-parameter family of solutions. Similarly, for $y'=\dfrac{1}{x^2}$, the one-parameter families of solutions are defined by $y_2=-\dfrac{1}{x}+C_2$ on the interval $(-\infty,0)\cup(0,\infty)$.
\end{eg}
\begin{df}{General Solution}
	Given the general form of the linear ODE $y'+p(x)y=f(x)$ if $p$ and $f$ are continuous on some open interval $(a,b)$ and there is a unique formula $y=y(x,c)$ and we have the following properties: 
	\begin{itemize}
		\item for each fixed $c$, the resulting function of $x$ is a solution of the ODE on $(a,b)$, and
		\item if $y$ is a solution of the ODE, then $y$ can be obtained by choosing the value of $c$ appropriately. 
	\end{itemize}
	The function $y=y(x,c)$ is called a \textit{general solution}. \par More generally, we can write an ODE as \[P_0(x)y'+P_1(x)y=F(x).\] In this case, the ODE has a general solution on any open interval in which $P_0,\ P_1,$ and $F$ are continuous and $P_0\neq0$.
\end{df}
\begin{df}{Initial Value Problem (IVP)}
	A differential equation with an initial condition.
\end{df}
\begin{eg}
	Let $a$ be a constant. Find the general solution of $y'-ay=0$ and solve the IVP $\begin{cases}y'-ay=0\\y(x_0)=y_0.\end{cases}$
	\begin{sol}
		Classification: First order, Linear, Homogeneous.\par Trivial Solution: $y=0$.\par General solution: \[\begin{aligned}\dv{y}{x}&=ay\\\int\dfrac{1}{y}\ \d y&=\int a\ \d x\\\ln\qty|y|&=ax+c\\y&=e^{ax+c}=Ae^{ax}.\end{aligned}\]\par\textit{This general solution includes the trivial solution.}\par IVP: Substitute $x=x_0$ and $y=y_0$: \[y_0=Ae^{ax_0}\quad\longrightarrow\quad A=y_0e^{-ax_0}\]\par So, \[y^\text{IVP}=y_0e^{-ax_0}e^{ax}=y_0e^{a(x-x_0)}.\]\par \textit{This IVP is a ``generic initial condition.'' We need more information on $x_0, y_0$ to get a more specific solution. }
	\end{sol}
\end{eg}

\subsection{Linear First Order ODEs}
\begin{thm}{}
	If $p$ is continuous on $(a,b)$, then the general solution of the homogeneous equation $y'+p(x)y=0$ on $(a,b)$ is given by \[y=ce^{-\int p(x)\ \d x}.\]
\end{thm}
\begin{prf}\par 
	(a). Substitute the solution formula to show that $y=ce^{-\int p(x)\ \d x}$ is a solution for any choice of $c$. \[y'=c\qty(-\int p(x)\ \d x)'e^{-\int p(x)\ \d x}=-cp(x)e^{-\int p(x)\ \d x}.\] Then, \[y'+p(x)y=-cp(x)e^{-\int p(x)\ \d x}+cp(x)e^{-\int p(x)\ \d x}=0.\] So, $y=ce^{-\int p(x)\ \d x}$ is a solution for any choice of $c$. $\pqde$\par
	(b). Want to show: any solution of $y'+p(x)y=0$ can be written as $y=ce^{-\int p(x)\ \d x}$. Note that $y=0$ is a trivial solution, so we assume $y\neq0$. \[\begin{aligned}y'+p(x)y&=0\\y'&=-p(x)y\\\dfrac{y'}{y}&=-p(x)\\\leadsto\int\dfrac{1}{y}\ \d y&=\int-p(x)\ \d x\\\ln\qty|y|&=-\int p(x)\ \d x\\y&=ce^{-\int p(x)\ \d x}.\end{aligned}\] Note that when $c=0$, $y=0$ is the trivial solution. So, any solution of $y'+p(x)y=0$ can be written as $y=ce^{-\int p(x)\ \d x}$.
\end{prf}
\begin{eg}
	Solve the IVP \[\begin{cases}xy'+y=0\\y(1)=3.\end{cases}\]
	\begin{sol}
		Note that $P_0(x)=x$ and $P_1(x)=1$, which are continuous on $\R$. Since we need $P_0(x)\neq0$, $x\neq0$. So the interval of validity is $\R\backslash\qty{0}$.\par 
		$\boxed{\text{Method 1: Separation of Variables}}$ \[y'=-\dfrac{y}{x}.\] Note that $y=0$ is a solution. Assume $y\neq0$. \[\begin{aligned}\dfrac{y'}{y}=-\dfrac{1}{x}\quad\leadsto\quad\int\dfrac{1}{y}\ \d y&=-\int\dfrac{1}{x}\ \d x+k\\\ln\qty|y|&=-\ln\qty|x|+k\\\qty|y|&=e^{k}\dfrac{1}{\qty|x|}\\y&=\dfrac{c}{x}\end{aligned}\]\par 
		$\boxed{\text{Method 2: Solution Formula}}$ By Theorem 1.2.1, \[y=ce^{-\int p(x)\ \d x}=ce^{-\int\frac{1}{x}\ \d x}=ce^{-\ln\qty|x|}=\dfrac{c}{x}.\]\par 
		$\boxed{\text{Solving the IVP}}$ Substitute $x=1$ and $y=3$: \[3=\dfrac{c}{1}\quad\longrightarrow\quad c=3.\] So, $y^\text{IVP}=\dfrac{3}{x}$.
	\end{sol}
\end{eg}
\begin{eg}
	Given the equation $(4+x^2)y'+2xy=4x$. Classify the equation and find the general solution $y=y(x,c)$.
	\begin{sol}
		This is a first order, linear, non-homogeneous differential equation.\par
		Note that $P_0(x)=4+x^2$, $P_1(x)=2x$, $F(x)=4x$, and $P_0\neq0\ \forall x\in\R$, so the interval of validity is $\R$. Also note that $\dsst\dv{x}\qty\big[4+x^2]=2x$, so the equation can be written as \[(4+x^2)\dv{y}{x}+\dv{x}\qty\big[4+x^2]y=4x.\] Using the product rule to re-write the LHS as \[\begin{aligned}\dv{x}\qty\big[(4+x^2)y]&=4x\\\int\dv{x}\qty\big[(4+x^2)y]\ \d x&=\int 4x\ \d x+c\\(4+x^2)y&=2x^2+c\\y&=\dfrac{2x^2+c}{4+x^2}.\end{aligned}\]
	\end{sol}
\end{eg}
\begin{eg}
	Given the equation $y'-2y=4-x$. Classify the equation and find the general solution $y=y(x,c)$.
	\begin{sol}
		This is a first order, linear, non-homogeneous differential equation.\par
		Since $P_0(x)=1$, $P_1(x)=-2y$, $F(x)=4-x$, and $P_0(x)\neq0\ \forall x\in\R$, the interval of validity is $\R$. Consider $\mu=\mu(x)\neq0$. Multiply both sides of the equation by $\mu(x)$: \begin{equation}\label{eq1}\mu(x)y'-2\mu(x)y=\mu(x)(4-x)\end{equation} To make the LHS a product rule, we need \[\dv{x}\qty\big[\mu(x)y(x)]=\mu'(x)y(x)+\mu(x)y'(x)=\mu(x)y'(x)-2\mu(x)y.\] So, we have $\mu'=-2\mu$, or $\mu'+2\mu=0,$ a first order, linear, homogeneous ODE. Solving this ODE, we get $\mu(x)=ce^{-2x}$. Since we only want one specific $\mu$ that would work, take $c=1$. So, $\mu(x)=e^{-2x}$. Substituting $\mu(x)=e^{-2x}$ to Eq. (\ref{eq1}): \[e^{-2x}y'-2e^{-2x}y=e^{-2x}(4-x),\quad\widetilde{P}_0=e^{-2x}\neq0,\ \widetilde{P}_1=-2e^{-2x}.\] Using the product rule: \[\begin{aligned}\dv{x}\qty[e^{-2x}y]&=4e^{-2x}-xe^{-2x}\\\int\dv{x}\qty[e^{-2x}y]\ \d x&=\int 4e^{-2x}-xe^{-2x}\ \d x+c\\e^{-2x}y&=\dfrac{1}{2}xe^{-2x}-\dfrac{7}{4}e^{-2x}+c\\y&=e^{2x}\qty(\dfrac{1}{2}xe^{-2x}-\dfrac{7}{4}e^{-2x}+c)\\&=\dfrac{1}{2}x-\dfrac{7}{4}+ce^{2x}.\end{aligned}\]
	\end{sol}
\end{eg}
\begin{thm}{Method of Integrating Factor}
	Given the first order linear differential equation $y'+p(x)y=f(x)$, with $p$ and $f$ both continuous on some interval $(a,b)$, \[y(x)=\dfrac{1}{\mu(x)}\qty[\int\mu(x)f(x)\ \d x+c]\] is the general solution to the equation, with \[\mu(x)=e^{\int p(x)\ \d x}.\] We call $\mu(x)$ the \textit{integrating factor}.
\end{thm}
\begin{prf}
	Consider $\mu=\mu(x)\neq0$. Multiplying the both sides of $y'+p(x)y=f(x)$ by $\mu$: \begin{equation}\label{eq2}\mu y'+p\mu y=\mu f.\end{equation} Impose $\mu y'+p\mu y=\dsst\dv{x}\qty\big[\mu y]$ to find $\mu=\mu(x)$: \[\begin{aligned}\mu y'+p\mu y&=\mu' y+\mu y'\\\mu'-p\mu&=0,&\text{first order, linear, homogeneous ODE}\\\mu(x)&=e^{\int p(x)\ \d x}, &\text{the integrating factor}\end{aligned}\] Substitute $\mu(x)=e^{\int p(x)\ \d x}$ into Eq. (\ref{eq2}): \[\begin{aligned}\dv{x}\qty\big[\mu y]&=\mu f\\\int\dv{x}\qty\big[\mu y]\ \d x&=\int\mu f\ \d x+c\\\mu y&=\int\mu f\ \d x+c\\y(x)&=\dfrac{1}{\mu(x)}\qty[\int\mu(x)f(x)\ \d x+c].\end{aligned}\]
\end{prf}
\begin{eg}
	Give the equation $y'+2y=x^3e^{-2x}$. Classify the equation and find the general solution $y=y(x,c)$.
	\begin{sol}
		It is a first order, linear, non-homogeneous ODE, with $p=2$ and $f=x^3e^{-2x}$. Let $\mu(x)$ be the integrating factor. Then, \[\mu(x)=e^{\int2\ \d x}=e^{2x}.\] So, by the method of integrating factor, we know \[\begin{aligned}\dv{x}\qty\Big[\mu(x)y]&=\mu(x)f(x)\\\int\dv{x}\qty\Big[e^{2x}y]\ \d{x}&=\int e^{2x}x^3e^{-2x}\ \d{x}+c\\e^{2x}y&=\int x^3\ \d{x}+c\\e^{2x}y&=\dfrac{1}{4}x^4+c\\y&=\dfrac{1}{4}x^4e^{-2x}+ce^{-2x}.\end{aligned}\]
	\end{sol}
\end{eg}
\begin{rmk}
	Re-examine the formula we derived from the method of integrating factor: \[y(x)=\dfrac{1}{\mu}\int f\mu\ \d x+\boxed{\dfrac{c}{\mu}}.\] The part being boxed, $\dfrac{c}{\mu}$, is independent from $f$ and is exactly $ce^{-\int p\ \d x}$ if we expand, which is the solution for a homogeneous differential equation. 
\end{rmk}
\begin{df}{Complementary Equation}
	The \textit{complementary equation} to a first order ODE $y'+py=f$ is the homogeneous part of it. i.e., $y'+py=0$.	
\end{df}
\begin{thm}{Method of Variation of Parameters}
	Given the first order linear differential equation $y'+p(x)y=f(x)$, with $p$ and $f$ both continuous on some interval $(a,b)$, \[y(x)=y_1(x)\qty[\int\dfrac{f(x)}{y_1(x)}\ \d x+c]\] is the general solution to the equation, where $y_1$ is a solution of the complementary euqation $y'+py=0$.
\end{thm}
\begin{prf}
	Call $y_1$ a solution of the complementary equation $y'+p(x)y=0$. Then, we want to find $y(x)=u(x)y_1(x)$, the general solution of $y'+p(x)y=f(x)$, where $u$ is an unknown function of $f$. Note that, by product rule, $y'(x)=u'y_1+uy_1'$. Then, the equation becomes \[\begin{aligned}(u'y_1+uy_1')+p(x)(uy_1)&=f(x)\\u'y_1+uy_1'+puy_1&=f\\y_1u'+\underbrace{(y_1'+py_1)}_{0}u&=f\\y_1u'&=f\implies u(x)=\int\dfrac{f(x)}{y_1(x)}\ \d x+c.\end{aligned}\] Therefore, the formula to find $y$ is given by \[y=y_1u=y_1(x)\qty[\int\dfrac{f(x)}{y_1(x)}\ \d{x}+c].\]
\end{prf}
\begin{rmk}
	The method of variation of parameters will be more useful when solving second or higher order differential equations.	
\end{rmk}
\begin{eg}
	Give the equation $y'+2y=x^3e^{-2x}$. Find the general solution $y=y(x,c)$ using the method of variation of parameters.
	\begin{sol}
		It is a first order, linear, non-homogeneous ODE, with $p=2$ and $f=x^3e^{-2x}$. Let $y_1$ be the solution of the complementary equation $y'+2y=0$. Then, $y_1(x)=e^{-\int2\ \d{x}}=e^{-2x}$. By the method of variation of parameters, suppose $y=uy_1$, where $u$ is an unknown function of $x$. Then, \[u(x)=\int\dfrac{f(x)}{y_1(x)}\ \d{x}+c=\int\dfrac{x^3e^{-2x}}{e^{-2x}}\ \d{x}+c=\int x^3\ \d{x}+c=\dfrac{1}{4}x^4+c.\] So, \[y=uy_1=e^{-2x}\qty(\dfrac{1}{4}x^4+c)=\dfrac{1}{4}x^4e^{-2x}+ce^{-2x}.\]
	\end{sol}
\end{eg}
\begin{thm}{Existence and Uniqueness Theorem}
	Suppose that $p=p(x)$ and $f=f(x)$ are continuous on $(a,b)$. Then, a general solution of $y'+p(x)y=f(x)$ on $(a,b)$ is  \[y(x)=y_1(x)\qty[\int\dfrac{f(x)}{y_1(x)}\ \d{x}+c],\]	 where $y_1(x)$ is a solution of the complementary equation (i.e., $y'+p(x)y=0$).\par 
	If $x_0$ is an arbitrary point in $(a,b)$ and $y_0$ is an arbitrary real number, then the initial value problem, \[y'+p(x)y=f(x),\quad y(x_0)=y_0\] has a unique solution on $(a,b)$.
\end{thm}

\subsection{Non-Linear First Order ODEs}
\begin{df}{General Forms}
	The general form of a non-linear first order ODE is given by \[y'=f\qty(x,y\qty(x)).\] If we take $M(x,y)=-f(x,y)$ and $N(x,y)=1$, we can also re-write the equation into \[M(x,y)+N(x,y)y'=0,\] or \[M(x,y)\ \d{x}+N(x,y)\ \d{y}=0.\]
\end{df}
\begin{df}{Separable Equations}
	If $M(x,y)=M(x)$ and $N(x,y)=N(y)$, then the ODE is called \textit{separable}.
\end{df}
\begin{thm}{Separation of Variables (SoV)}
	Consider the non-linear first order ODE $M(x,y)+N(x,y)y'=0$, with $M(x,y)=M(x)$ and $N(x,y)=N(y)$. Then we can find an implicit solution of the ODE in the form of \[F(x,y)=c,\] where $F(x,y)$ is a function of $x$ and $y$ and \[F(x,y)=\int M(x)\ \d{x}+\int N(y)\ \d{y}.\]
\end{thm}
\begin{prf}
	Let $H_1'(x)=M(x)$ and $H_2'(y)=N(y)$. Then, the equation becomes \[\begin{aligned}H_1'(x)+H_2'(y)y'&=0\\\dv{x}\qty\Big[H_1(x)]+\dv{y}\qty\Big[H_2(y)]\dv{y}{x}&=0\end{aligned}\] By using the chain rule, $\dsst\dv{y}\qty\Big[H_2(y)]\dv{y}{x}=\dv{x}\qty\Big[H_2\qty(y(x))].$ So, the equation becomes \[\begin{aligned}\dv{x}\qty\Big[H_1(x)]+\dv{x}\qty\Big[H_2\qty(y(x))]&=0\\\dv{x}\qty\Big[H_1(x)+H_2(y)]&=0\\H_1(x)+H_2(y)&=c\\\int M(x)\ \dv{x}+\int N(y)\ \d{y}&=c\\F(x,y)&=c\end{aligned}\]
\end{prf}
\begin{eg}
	Given the equation $\dsst y'=\dfrac{x^2}{1-y^2}$. Classify the differential equation and find the general solution.
	\begin{sol}
		It is a first order, non-linear ODE. Since $y'=\dfrac{x^2}{1-y^2}$, so we have $1-y^2\neq0$. That is, $y^2\neq1,$ or $y\neq\pm1$. Using the separation of variables (SoV), we have \[\begin{aligned}(1-y^2)y'&=x^2\\\int 1-y^2\ \d{y}&=\int x^2\ \d{x}\\y-\dfrac{1}{3}y^3&=\dfrac{1}{3}x^3+c\\y-\dfrac{1}{3}y^3-\dfrac{1}{3}x^3&=c\\3y-y^3-x^3&=c\end{aligned}\]
	\end{sol}
\end{eg}
\begin{eg}
	Given the equation $\dsst y'=\dfrac{(y-3)\cos{x}}{1+2y^2}$. Classify the equation and find the general solution.
	\begin{sol}
		It is a first order, non-linear ODE. Since $1+2y^2\neq0\quad\forall y\in\R$. Note that if we take $y-3=0$, we get $y=3$, a constant solution to the differential equation. Now, assume $y\neq3$. Then, use SoV: \[\int\dfrac{1+2y^2}{y-3}\ \d{y}=\int\cos{x}\ \d{x}+c=\sin{x}+c.\] Set $t=y-3$, $\d{t}=\d{y}$. So, $y=t+3$ and $y^2=(t+3)^2$. Then, \[\begin{aligned}\int\dfrac{1+2y^2}{y-3}\ \d{y}=\int\dfrac{1+2(t+3)^2}{t}\ \d{t}&=\int\dfrac{1+2t^2+12t+18}{t}\ \d{t}\\&=\int\dfrac{19}{t}+12+2t\ \d{t}\\&=19\ln\qty|t|+12t+t^2\\&=19\ln\qty|y-3|+12(y-3)+(y-3)^2\\&=19\ln\qty|y-3|+6y+y^2-27.\end{aligned}\] So, \[19\ln\qty|y-3|+y^2+6y-27-\sin{x}=c\]
	\end{sol}
\end{eg}
\begin{eg}
	Give the equation $y'=\dfrac{1}{2}x\qty(1-y^2)$. Classify the equation and find the general solution. 
	\begin{sol}
		It is a first order, non-linear ODE. Notice that we have the constant solutions when we take $1-y^2=0$, or $y=\pm1$. Now, assume $y\neq\pm1$. Using SoV: \[\int\dfrac{2}{1-y^2}\ \d{y}=\int x\ \d{x}+c=\dfrac{1}{2}x^2+c.\]	Note that $\dfrac{2}{1-y^2}=\dfrac{2}{(1-y)(1+y)}$. Use partial fractions. Assume \[\dfrac{2}{(1-y)(1+y)}=\dfrac{A}{1-y}+\dfrac{B}{1+y}.\] Then, we get $A(1+y)+B(1-y)=2$. That is, $(A+B)+(A-B)y=2$. Attempting to solve the system of equations $\begin{cases}A-B=0\\A+B=2\end{cases}$, then we get $A=B=1$.Therefore, \[\dfrac{2}{1-y^2}=\dfrac{1}{1-y}+\dfrac{1}{1+y}.\] Then, \[\begin{aligned}\int\dfrac{1}{1-y}+\dfrac{1}{1+y}\ \d{y}&=\dfrac{1}{2}x^2+c\\-\ln\qty|1-y|+\ln\qty|1+y|&=\dfrac{1}{2}x^2+c\\\ln\qty|1-y|-\ln\qty|1+y|&=-\dfrac{1}{2}x^2+c\\\ln\qty|\dfrac{1-y}{1+y}|&=-\dfrac{1}{2}x^2+c\\\qty|\dfrac{y-1}{y+1}|&=e^{-\frac{1}{2}x^2+c}=e^{-\frac{1}{2}x^2}e^c\\\dfrac{y-1}{y+1}&=c_2e^{-\frac{1}{2}x^2}\\y-1&=(y+1)c_2e^{-\frac{1}{2}x^2}\\(1-c_2e^{-\frac{1}{2}x^2})y&=1+c_2e^{-\frac{1}{2}x^2}\\y&=\dfrac{1+c_2e^{-\frac{1}{2}x^2}}{1-c_2e^{-\frac{1}{2}x^2}}\end{aligned}\] The value of $c_2$ si chosen according to the sign of $\dfrac{y-1}{y+1}.$
	\end{sol}
\end{eg}
\begin{eg}{}
	Given the equation $y'=\dfrac{3x^2+4x+2}{2(y-1)},$ with $y(0)=1$. Classify the equation, find the general solution, and solve the IVP. 
	\begin{sol}
		It is a first order, nonlinear, separable ODE. Note that $y-1\neq0$, so $y\neq1$. Assume $y\neq1,$ use SoV: \[\begin{aligned}\int2(y-1)\ \d{y}&=\int3x^2+4x+2\ \d{x}+c\\(y-1)^2&=x^3+2x^2+2x+c\\y^2-2y+1&=x^3+2x^2+2x+c\\y^2-2y&=x^3+2x^2+2x+c.\end{aligned}\] Substitute $y=-1$ when $x=0$: \[1+2=c\implies c=3.\] So, \[\begin{aligned}y^2-2y&=x^3+2x^2+2x+3,\quad y\neq1\\y^2-2y+1&=x^3+2x^2+2x+4\\(y-1)^2&=x^3+2x^2+2x+4\\y-1&=\pm\sqrt{x^3+2x^2+2x+4}\\ y&=1\pm\sqrt{x^3+2x^2+2x+4}.\end{aligned}\] If $y=-1$ and $x=0$: $-1=1\pm\sqrt{4}=1\pm2$. So, it must be that $-1=1-2$. So, \[y^\text{IVP}=1-\sqrt{x^3+2x^2+2x+4}.\] Note that now we have another condition for $x$: \[\begin{aligned}x^3+2x^2+2x+4&\geq0\\(x+2)(x^2+2)&\geq0\\x+2&\geq0\\x&\geq-2\end{aligned}\] So, \[y^\text{IVP}=1-\sqrt{x^3+2x^2+2x+4},\text{ with }y\neq1\text{ and }x\geq-2.\]
	\end{sol}
\end{eg}
\begin{eg}{}
	Solve the IVP $\begin{cases}y'=\sqrt[3]{y}=y^\frac{1}{3}\\y(0)=0\end{cases}$. 
	\begin{sol}
		It is a first order, nonlinear, separable ODE. The initial interval of validity: $x\in\R$ and $y\in\R$. Note that if $y=0$, there is a constant solution. Assume $y\neq0$, use $SoV$: \[\begin{aligned}\int y^{-\frac{1}{3}}\ \d{y}&=\int\ \d{x}+c\\\dfrac{3}{2}y^\frac{2}{3}&=x+c\\y^\frac{2}{3}&=\dfrac{2}{3}x+c\\y&=\pm\qty(\dfrac{2}{3}x+c)^\frac{3}{2}\end{aligned}\] Substitute $y(0)=0$: \[0=0+c\implies c=0.\] So, \[y^\text{IVP}=\pm\qty(\dfrac{2}{3}x)^\frac{3}{2}.\]
	\end{sol}
\end{eg}
\begin{thm}{Existence and Uniqueness of Solutions to Nonlinear ODEs}
	Consider the IVP \[y'=f(x,y(x))\quad\text{with }y(x_0)=y_0.\]
	\begin{itemize}
		\item If $f$ is continuous on an open rectangle $R\qty{a<x<b, c<y<d}$ that contains $(x_0,y_0)$, then the IVP has \textit{at least} one solution on some open subinterval of $(a,b)$ that contains $x_0$.
		\item If both $f$ and $\dsst\pdv{f}{y}$ are continuous on $R$, then the IVP has a \textit{unique} solution on some open subinterval of $(a,b)$ that contains $x_0$.
	\end{itemize}	
\end{thm}
\begin{eg}
	In the IVP above (Example 1.3.8), $f(x,y)=y^\frac{1}{3},$ and so $\dsst\pdv{f}{y}=\dfrac{1}{3}y^{-\frac{2}{3}}$, which is not continuous at $y=0$. So, the IVP $\nexists$ a unique solution on the interval given: $R=\qty{x\in\R,y\in\R}.$	
\end{eg}



\section{Second Order ODEs}

\section{System of ODEs}
\end{document}
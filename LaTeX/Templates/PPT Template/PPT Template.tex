% UTF-8 encoding
\documentclass[9pt, dvipsnames]{beamer} 

% Beamer 设置
\usetheme[secheader]{Boadilla} % 使用的 Beamer 主题: Boadilla, CambridgeUS, Madrid
\usecolortheme{beaver} % 使用的 Beamer 颜色:beaver, orchid, rose, 
% 字体设置
\usefonttheme{serif} % professional 字体: serif, professionalfonts, structurebold

% 其他 Package
\usepackage{amssymb, amsmath, amsthm, amsfonts, epsfig, mathtools, times, verbatim, anyfontsize, subfigure, graphicx, tabularx, tikz, mathptmx}
\usepackage[export]{adjustbox}
\setbeamertemplate{caption}[numbered]
\newcounter{saveenumi}
\resetcounteronoverlays{saveenumi}
\usepackage[multidot]{grffile} % 允许文件名带多个点
%\usepackage{ctex} % xelatex 中文

\usepackage[T1]{fontenc}

% 使用背景: \pgfdeclareimage[height=\paperheight,width=\paperwidth]{bgimage}{lzu.jpg} \usebackgroundtemplate{\tikz\node[opacity=0.1,inner sep=0]{\pgfuseimage{bgimage}};}


\title{Example Code of Beamer} % 标题
\author[Your Name]{Your Name} % 作者
\date{Jan 5, 2019} % 如果 Date 参数为空,自动显示当前日期

\begin{document}
\everymath{\displaystyle}

% 标题页
\begin{frame}
    \titlepage % 根据上面信息生成标题
\end{frame}

\begin{frame}
    \frametitle{\textbf{Index}}
    \tableofcontents % 生成目录(如果为空,请编译两次)
\end{frame}

\section{Introduction}\label{sec:introduction}
\begin{frame}
\frametitle{\textbf{Introduction}}
\begin{block}{Block Title 1}
\begin{itemize}
    \item Lorem ipsum dolor sit amet, consectetur adipiscing elit, sed do eiusmod tempor incididunt ut labore . {\scriptsize \color{red} Author A, Author B 2018 }
    \item Ut enim ad minim veniam, quis nostrud exercitation ullamco laboris nisi ut aliquip ex ea commodo consequat.
    \item Duis aute irure dolor in reprehenderit in voluptate velit esse cillum dolore eu fugiat nulla pariatur.  {\scriptsize \color{red} Author A, Author B 2016 }
\end{itemize}
\end{block}
    
\begin{exampleblock}{Block Title 2}
\begin{enumerate}
    \item Lorem ipsum dolor sit amet, consectetur adipiscing elit, sed do eiusmod tempor incididunt ut labore . {\scriptsize \color{red} Author A, Author B 2018 }
    \item Ut enim ad minim veniam, quis nostrud exercitation ullamco laboris nisi ut aliquip ex ea commodo consequat.
    \item Duis aute irure dolor in reprehenderit in voluptate velit esse cillum dolore eu fugiat nulla pariatur.  {\scriptsize \color{red} Author A, Author B 2016 }
\end{enumerate}
\end{exampleblock}
\end{frame}

\begin{frame}
\frametitle{\textbf{Example of subfigure}}
\centering
Idea A $\Longleftrightarrow$ Idea B
\vskip 2em
\begin{figure}
    \subfigure[Image Caption]{\includegraphics[width=0.4\linewidth]{example-image}}
    \subfigure[Image Caption]{\includegraphics[width=0.4\linewidth]{example-image}}
    \caption{This is a caption}
\end{figure}
\end{frame}

\begin{frame}
\frametitle{\textbf{Black hole}}
\begin{exampleblock}{The metric and the electromagnetic field of the spherically symmetric solution}
\begin{align}
    ds^2&=-fdt^2+\frac{dr^2}{f}+r^2 d\Omega_2^2\,,\label{BImetric}\\
    F&=Edt\wedge dr\label{BIE}\,, \quad E=\frac{Q}{\sqrt{r^4+Q^2/b^2}}\,.
\end{align}
\end{exampleblock}
where
\begin{equation}
\begin{split}
    f=&1-\frac{2M}{r}+\frac{r^2}{l^2}+\frac{2b^2}{r}\int_r^\infty \Bigl(\sqrt{r^4+\frac{Q^2}{b^2}}-r^2\Bigr)dr\nonumber\\
    =&1-\frac{2M}{r}+\frac{r^2}{l^2}+\frac{2b^2 r^2}{3}\Bigl(1-\sqrt{1+\frac{Q^2}{b^2 r^4}}\Bigr)\nonumber\\
    &+\frac{4Q^2}{3r^2}\,{}_2 F_1\left(\frac{1}{4},\frac{1}{2}; \frac{5}{4};-\frac{Q^2}{b^2 r^4}\right)\,,
\end{split}
\end{equation}
and $_2 F_1$ is the hypergeometry function, $M$ and $Q$ stand for black hole mass and charge. $d\Omega$ is the unit sphere on $S^2$.
\end{frame}

\section{Content}\label{sec:content}
\begin{frame}
\frametitle{\textbf{Content}}
\begin{block}{Mass $M$}
\begin{equation}
    f(r_h) = 0 \Longrightarrow M = \frac{T}{v}-\frac{1-\sqrt{\frac{16}{v^4}+1}}{4 \pi }-\frac{1}{2 \pi  v^2}
\end{equation}
\end{block}
    
\begin{exampleblock}{Hawking temperature $T$}
\begin{equation}\label{eq:T}
    T = f^{\prime}(r_+) / 4 \pi = \frac{1}{4\pi r_+}\!\left[1\!+\!\frac{3r_+^2}{l^2}\!+\!2b^2 r_+^2\Bigl(1\!-\!\sqrt{1\!+\!\frac{Q^2}{b^2 r_+^4}}\Bigr)\right]
\end{equation}
\end{exampleblock}
    
\begin{alertblock}{Electric potential $\Phi$}
\begin{equation}
    \Phi=\int_{r_+}^\infty E dr
    =\frac{Q}{r_+}\,{}_2 F_1\!\left(\frac{1}{4},\frac{1}{2};\frac{5}{4};-\frac{Q^2}{b^2 r_+^4}\right)\,.
\end{equation}
\end{alertblock}
    
The corresponding entropy is $S = \pi r_+^2$, The specific volume $v = 2 r_+ l_P^2$ and corresponding pressure $P = - \frac{\Lambda}{8 \pi} = \frac{3}{8 \pi l^2}$
\end{frame}

\section{Conclusion}\label{sec:conclusion}
\begin{frame}
\frametitle{Conclusion}
\begin{block}{Conclusion 1}
    Lorem ipsum dolor sit amet, consectetur adipiscing elit, sed do eiusmod tempor incididunt ut labore et dolore magna aliqua.
    Ut enim ad minim veniam, quis nostrud exercitation ullamco laboris nisi ut aliquip ex ea commodo consequat.
\end{block}

\begin{exampleblock}{Conclusion 2}
    Lorem ipsum dolor sit amet, consectetur adipiscing elit, sed do eiusmod tempor incididunt ut labore et dolore magna aliqua.
    Ut enim ad minim veniam, quis nostrud exercitation ullamco laboris nisi ut aliquip ex ea commodo consequat. 
\end{exampleblock}
\end{frame}

\begin{frame}[noframenumbering]
\centering
\fontsize{40}{50}\selectfont Thank You!
\end{frame}

\end{document}
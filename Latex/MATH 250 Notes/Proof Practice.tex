\documentclass[11pt,letter]{article}
\usepackage{ntheorem}
\usepackage{amssymb, amsmath, amsfonts, epsfig, graphicx, dsfont, bbm, url, color, setspace, multirow, pinlabel, pifont, framed, physics, mathtools, hyperref}
\usepackage{fancyhdr, geometry}

% Page Setting
\geometry{left=2cm, right=2cm, bottom=2cm, top=2cm}
\pagestyle{fancy}
\linespread{1.5}

% Head & Foot Setting
\fancyhead[L]{} 
\fancyhead[R]{}
\chead{\textbf{Practice Proofs}} 
\cfoot{\textbf{Page \thepage\ of \pageref{LastPage}}}
\fancyfoot[L]{} 
\fancyfoot[R]{} 
\renewcommand{\headrulewidth}{0.5pt}
%\renewcommand{\footrulewidth}{0.5pt}

% Defining Environments
\newcounter{nq}[section]
\setcounter{nq}{0}
\newcounter{np}[section]
\setcounter{np}{0}
\newtheorem{thm}{Theorem}[section]
\newtheorem{df}{Definition}[section]
\newtheorem{eg}{Example}
\newtheorem{ax}{Axiom}[section]
\newtheorem{prop}{Proposition}[section]
\newtheorem{lem}{Lemma}[section]
\newtheorem{cor}{corollary}[section]
\newtheorem*{rmk}{\indent Remark}
\newtheorem*{ext}{\indent Extension}
\newenvironment*{p}{\par\noindent\textbf{\textit{Proof \stepcounter{np}\thenp. }}\par}{\par\hfill $\blacksquare$\par}
\newenvironment*{dis}{\par\noindent\textbf{\textit{Disproof \stepcounter{np}\thenp. }}\par}{\par\hfill $\blacksquare$\par}
\newenvironment*{q}[1]{\noindent\emph{\thesection.\stepcounter{nq}\thenq$\quad $ #1}\par\noindent\texttt}{}
\newenvironment*{clm}{\par\noindent\textbf{Claim. }}{\par}
\newenvironment*{counter}{\par\noindent\textbf{\textit{Counterexample\stepcounter{np}\thenp. }}\par}{\par\hfill $\square$\par}
\newenvironment*{ans}{\par\noindent\textbf{\textit{Answer \stepcounter{np}\thenp. }}\par}{\par\hfill $\square$\par}


% Defining Symbols
\def\Z{{\mathbb{Z}}}
\def\R{{\mathbb{R}}}
\def\C{{\mathbb{C}}}
\def\Q{{\mathbb{Q}}}
\def\N{{\mathbb{N}}}
\def\E{{\mathbb{E}}}
\def\Zp{{\Z^+}}
\def\emptyset{\varnothing}
\def\epsilon{\varepsilon}
\def\pwer{{\mathcal{P}}}
\def\qed{\rightline{$\blacksquare$}}
\def\st{\emph{ s.t. }}
\def\fs{\emph{ f.s. }}
\def\Iff{\emph{ iff }}
\def\dsst{\displaystyle}
\def\f{f^{-1}}
\def\of{\circ}

% Title Information
\title{Foundations of Mathematics -- \textbf{Proof Practice}}
\author{Jiuru Lyu}
\date{\today}

\begin{document}

\section{Statements}
\begin{framed}\begin{q}
	{Class Handout, Chapter 1.3, Implications.}
	{Let $a,\ b,\ $and $c$ be integers, with $a$ and $b$ non-zero. If $(ab)\mid(ac)$, then $b\mid c.$}
\end{q}\end{framed}
\begin{p}
	Let	$a,b,c\in\Z$ with $a\neq0$ and $b\neq0$. Suppose $(ab)\mid(ac)$. Then $\exists k\in\Z\st ac=(ab)k$. Divide both sides of the equation by $a$: \[c=bk.\] Since $k\in\Z$, by definition of divides, $b\mid c$.
\end{p}

\begin{framed}\begin{q}
	{Class Handout, Chapter 1.4, Contrapositive and Converse}
	{Prove that for all real numbers $a$ and $b$, if $a\in\Q$ and $ab\notin\Q$, then $n\notin Q$.}	
\end{q}\end{framed}
\begin{p}
	Let $a,b\in\Q$. Assume for the sake of contradiction that if $a\in\Q$ and $ab\notin\Q$, we have $b\in\Q$. Then, $\exists p,q,m,n\in\Z\st a=\dfrac{m}{n}$ and $b=\dfrac{p}{q}$. Hence, \[ab=\frac{m}{n}\cdot\frac{p}{q}=\frac{mp}{nq}\] As $mp, nq\in\Z$, $ab\in\Q.$\begin{center}$\divideontimes$ This contradicts with the fact that $ab\notin\Q$.\end{center} So, $b$ must not be rational. 
\end{p}

\begin{framed}\begin{q}
{Chapter 1.1 \# 7(c)}
{Prove the square of an even integer is divisible by 4.}	
\end{q}\end{framed}
\begin{p}
	Suppose $x\in\Z$ is even. Then $\exists k\in\Z\st x=2k.$ Then, $x^2=(2k)^2=4k^2.$ Since $k^2\in\Z$, we have $4\mid 4k^2$.
\end{p}

\begin{thm}[Archimedean Principle]\label{AP}
	For every real number $x$, there is an integer $n$, such that $n>x$.	
\end{thm}

\begin{framed}\begin{q}
	{Chapter 1.1 \# 11}
	{For every positive real number $\varepsilon$, there exists a positive integer $N$ such that $\dfrac{1}{n}<\varepsilon$ for all $n\geq N.$}
\end{q}\end{framed}
\begin{p}
	Suppose $\varepsilon\in\R$ and $\varepsilon>0.$ Since $\varepsilon\in\R$, we have $\dfrac{1}{n}\in\R.$ Then, by Archimedean Principle, $\exists n\in\Z\st n>\dfrac{1}{\varepsilon}.$ Hence, $n\varepsilon>1$ or $\varepsilon>\dfrac{1}{n}.$\par 
	Suppose $N\in\Z\st N=\left\lceil\dfrac{1}{\varepsilon}\right\rceil$, \textit{where $\left\lceil\dfrac{1}{\varepsilon}\right\rceil$ means the integer greater to $\dfrac{1}{\varepsilon}$ if $\dfrac{1}{\varepsilon}\notin\Z$, and the integer equals to $\dfrac{1}{\varepsilon}$ if $\dfrac{1}{\varepsilon}\in\Z.$} Hence, $N\geq\dfrac{1}{\varepsilon}.$ As $n>\dfrac{1}{\varepsilon},$ we have $n\geq N$
\end{p}

\begin{framed}\begin{q}
	{Chapter 1.1 \# 12}
	{Use the Archimedean Principle (Theorem \ref{AP}) to prove if $x$ is a real number, then there exists a positive integer $n$ such that $-n<x<n.$}
\end{q}\end{framed}
\begin{p}
	Suppose $x\in\R.$\par 
	$\boxed{\text{Case }1}$ If $x>0$, then $-x<0$ (i.e., $-x<0<x$). By the Archimedean Principle, $\exists n\in\Z\st n>x.$ Multiply $(-1)$ on both sides of the inequality: \[-n<-x\] As $-x<0<x$, \[-n<-x<0<x<n,\] which means $-n<x<n,$ and $n$ is positive.\par 
	$\boxed{\text{Case }2}$ If $x<0,$ then $-x>0$ (i.e., $-x>0>x$) Since $x\in\R$, we have $-x\in\R$. By the Archimedean Principle, $\exists n\in\Z\st n>-x.$ Multiply $(-1)$ on both sides of the inequality: \[-n<x\] As $x<0<-x,$ \[-n<x<0<-x<n,\] which means $-n<x<n,$ and $n$ is positive. In all cases, we have proven that $x\in\R\implies\exists n\in\Z,\ n>0\st-n<x<n.$
\end{p}

\begin{framed}\begin{q}
	{Chapter 1.1 \# 13}
	{Prove that if $x$ is a positive real number, then there exists a positive integer $n$ such that $\dfrac{1}{n}<x<n$.}
\end{q}\end{framed}
\begin{p}
	Suppose $x\in\R,\ x>0$\par 
	$\boxed{\text{Case }1}$ If $0<x\leq1,$ then $\dfrac{1}{x}\geq1.$ Hence, $x\leq1\leq\dfrac{1}{x}.$ As $x\in\R,$ $\dfrac{1}{x}\in\R$, then by the Archimedean Principle (Theorem \ref{AP}): \[\exists n\in\Z\st n>\dfrac{1}{x}.\] Hence, $nx>1$ or $x>\dfrac{1}{n}.$ As $x\leq\dfrac{1}{x},$ $n>\dfrac{1}{x}$, and $x>\dfrac{1}{n}$, we have \[\frac{1}{n}<x<n.\]\par
	$\boxed{\text{Case }2}$ If $x>1$, then $0<\dfrac{1}{x}<1.$ Hence, $\dfrac{1}{x}<1<x.$ As $x\in\R$, by the Archimedean Principle: \[\exists n\in\Z\st n>x>0\] Hence, $\dfrac{1}{n}<\dfrac{1}{x}$. As $\dfrac{1}{x}<x$, $\dfrac{1}{n}<\dfrac{1}{x},$ and $n>x,$ we have \[\dfrac{1}{n}<x<n\]\par 
	In all cases, we proven that $x\in\R,\ x>0\implies\exists n\in\Z, n>0\st\dfrac{1}{n}<x<n.$
\end{p}

\begin{framed}\begin{q}
	{Handout Chapter 1.4-2 More Contradictions and Equivelance}
	{There are no positive integer solutions to the equation $x^2-y^2=10$.}
\end{q}\end{framed}
\begin{p}
	Assume for the sake of contradiction that there are positive integer solutions to the equation $x^2-y^2=10.$ Suppose $\exists x,y\in\Z$ and $x>0,\ y>0\st x^2-y^2=10.$ Then, we have $x^2=10+y^2.$ Since $x>0,\ x^2>0,$ we have $10+y^2>0.$ Then, $y^2>-10.$\begin{center}$\divideontimes$ This contradicts with the fact that $y^2\geq0$ if $y\in\Z.$\end{center} So, our assumption is wrong. There must be no positive integer solutions to the equation $x^2-y^2=10.$
\end{p}

\begin{framed}\begin{q}
	{Handout Chapter 1.4-2 More Contradictions and Equivelance}
	{Show that if $a\in\Q$ and $b\in\Q'$, then $a+b\in\Q'$}
	\begin{rmk}
		The notation $\Q$ means the set for rational numbers, and $\Q'$ means the set for irrational numbers.
	\end{rmk}
\end{q}\end{framed}
\begin{p}
	Suppose $a\in\Q$ and $b\in\Q'$ Assume for the sake of contradiction that $a+b\in\Q$. Then, $\exists m,n,p,q\in\Z$ such that $a=\dfrac{m}{n}$ and $a+b=\dfrac{p}{q}.$ Then, \[b=\frac{p}{q}-a=\frac{p}{q}-\frac{m}{n}=\frac{pn-mq}{qn}\in\Q\] Since $pn-mq\in\Q$ and $qn\in\Z$, we have $b=\dfrac{pn-mq}{qn}\in\Q$.\par\begin{center}$\divideontimes$ This contradicts with the fact that $b\in\Q'$. \end{center} So, $a+b$ must be irrational. 
\end{p}

\begin{framed}\begin{q}
	{Handout Chapter 1.4-2 More Contradictions and Equivalence}
	{If $n\in\N$ and $2^n-1$ is prime, then $n$ is prime.}
\end{q}\end{framed}
\begin{p}
	We will prove the contrapositive: if $n$ is not prime, then $2^n-1$ is not prime. Suppose $n$ is not prime. Then, $\exists a,b\in\Z$ with $1<a,b<n\st n=ab.$ Then, $2^n-1=2^{ab}=\left(2^a\right)^b-1.$ Notice that for $x^w-1$, by polynomial long division, have \[x^w-1=(x-1)\left(x^{w-1}+x^{w-2}+\cdots+1\right),\] Substitute $x=2^a$ and $w=b$, we have \[2^n-1=\left(2^a-1\right)\left[\left(2^a\right)^{b-1}+\left(2^a\right)^{b-2}+\cdots+1\right].\] Since $\left(2^a-1\right)\in\Z$ and $\left[\left(2^a\right)^{b-1}+\left(2^a\right)^{b-2}+\cdots+1\right]\in\Z,$ we see that $2^n-1$ is not prime. 
\end{p}

\begin{framed}\begin{q}
	{Exam 1 Review 1-b-i}
	{Prove that $\qty[P\wedge\qty(P\implies Q)]\implies Q$.}
\end{q}\end{framed}
\hspace*{\fill}\newline
\hspace*{\fill}
\begin{p}
	\begin{center}\begin{tabular}{c|c|c|c|c}
		$P$&$Q$&$P\Rightarrow Q$&$P\wedge(P\Rightarrow Q)$&$\qty[P\wedge(P\Rightarrow Q)]\implies P$\\
		\hline
		T&T&T&T&T\\
		T&F&F&F&T\\
		F&T&T&F&T\\
		F&F&T&F&T
	\end{tabular}\end{center}
\end{p}

\begin{framed}\begin{q}
	{Exam 1 Review 1-b-ii}
	{Prove that $\qty[Q\wedge\qty(P\implies Q)]\implies P$.}
\end{q}\end{framed}
\begin{p}
	\begin{center}\begin{tabular}{c|c|c|c|c}
		$P$&$Q$&$P\Rightarrow Q$&$Q\wedge(P\Rightarrow Q)$&$\qty[Q\wedge(P\Rightarrow Q)]\implies Q$\\
		\hline
		T&T&T&T&T\\
		T&F&F&F&T\\
		F&T&T&T&T\\
		F&F&T&F&T
	\end{tabular}\end{center}
\end{p}

\begin{framed}\begin{q}
	{Exam 1 Review 2-a}
	{Given statements $P$ and $Q$, prove $\neg(P\vee Q)\equiv\neg P\wedge\neg Q$.}
\end{q}\end{framed}
\begin{p}
	\begin{center}\begin{tabular}{c|c|c|c|c|c|c}
		$P$&$Q$&$P\vee Q$&$\neg(P\vee Q)$&$\neg P$&$\neg Q$&$\neg P\wedge\neg Q$\\
		\hline
		T&T&T&F&F&F&F\\
		T&F&T&F&F&T&F\\
		F&T&T&F&T&F&F\\
		F&F&F&T&T&T&T
	\end{tabular}\end{center}
\end{p}

\begin{framed}\begin{q}
	{Exam 1 Review 2-b}
	{There is no smallest integer.}
\end{q}\end{framed}
\begin{p}
	Assume for the sake of contradiction that there exists a smallest integer $n$. Hence, $\forall x\in\Z,$ we have $x\geq n.$ Notice that if $n>0,$ we have $0\in\Z$ and $0<n.$ Hence, $n=0$ cannot be the smallest integer ($\divideontimes$) Therefore, $n$ most be smaller than $0$. Suppose $m=-n$. Since $n\in\Z$, $m=-n\in\Z\in\R$ By the Archimedean Principle (Theorem \ref{AP}), $\exists k\in\Z\st k>m$. Hence, $k>-n.$ Multiply $(-1)$ on both sides of the inequality: \[-k<n.\] As $k\in\Z$, $-k\in\Z$. Then $\exists -k\in\Z\st -k<n.$\begin{center}$\divideontimes$ This contradicts with our assumption that $n$ is the smallest integer. \end{center} Hence, our assumption must be wrong. There is no smallest integer. 
\end{p}

\begin{framed}\begin{q}
	{Exam 1 Review 2-c}
	{The number $\log_23$ is irrational.}
\end{q}\end{framed}
\begin{p}
	Assume for the sake of contradiction that $\log_2{3}$ is irrational. By definition, $\exists p,qin\Z,$ with $q\neq0\st\log_2{3}=\dfrac{p}{q}.$ Observe that $\log_2{3}\neq0$. Then $p\neq0$ as well. By definition of logarithm, \[\begin{aligned}2^{p/q}&=3\\\qty(2^p)^{1/q}&=3\end{aligned}\] Raise two sides of the equation to the power of $q$: \[2^p=3^q\] As $p\neq0$ and $q\neq0,$ $2^p$ and $3^q$ are not $1\ \forall p,q\in\Z.$ Hence, $2^p$ is even $\forall p\in\Z$ and $3^q$ is odd $\forall q\in\Z.$\begin{center}$\divideontimes$ This contradicts with the fact that an even number cannot equal to an odd number. \end{center} Hence, our assumption is wront. The number $\log_23$, then, must be irrational.
\end{p}

\begin{framed}\begin{q}
	{Exam 1 Review 2-d}
	{There is a rational number $a$ and an irrational number $b$ such that $a^b$ is rational.}
\end{q}\end{framed}
\begin{p}
	Observe that $1$ is a rational number and $\pi$ is an irrational number.  Suppose $a=1$ and $b=\pi$, we have $a^b=a^\pi=1,$ which is irrational.	
\end{p}
\begin{p}
	Recall that we have proven in the previous proof, we have proven that $\log_23$ is an irrational number. Recall the definition of logarithm and exponents, we have \[2^{\log_23}=3\] Hence, we find a pair of $a$ and $b$ that  satisfies the requirement. 
\end{p}

\begin{framed}\begin{q}
	{Exam 1 Review 2-e}
	{For all integers $n$, the number $n+n^2+n^3+n^4$ is even.}
\end{q}\end{framed}
\begin{p}
	Suppose $n\in\Z$.\par
	$\boxed{\text{Case }1}$ If $n$ is even.Suppose $n=2k\fs k\in\Z$. Then,\[\begin{aligned}n+n^2+n^3+n^4&=(2k)+(2k)^2+(2k)^3+(2k)^4\\&=2k+4k^2+8k^3+16k^4\\&=2(k+2k^2+4k^3+8k^4)\end{aligned}\] Since $(k+2k^2+4k^3+8k^4)\in\Z,$ we have $2(k+2k^2+4k^3+8k^4)$ is even. Hence, $n+n^2+n^3+n^4$ is even when $n$ is even.\par
	$\boxed{\text{Case }2}$ If $n$ is odd. Suppose $n=2k+1\fs k\in\Z.$ Then, \[\begin{aligned}n+n^2+n^3+n^4&=(2k+1)+(2k+1)^2+(2k+1)^3+(2k+1)^4\\&=2k+1+4k^2+4k+1+8k^3+12k^2+6k+1+16k^4+32k^3+24k^2+8k+1\\&=16k^4+40k^3+40k^2+20k+4\\&=2(8k^4+20k^3+20k^2+10k+2)\end{aligned}\] Since $(8k^4+20k^3+20k^2+10k+2)\in\Z$, we have $2(8k^4+20k^3+20k^2+10k+2)$ is even. Hence, $n+n^2+n^3+n^4$ is even when $n$ is odd.\par 
	Since integers can either be even or odd, and we have proven $n+n^2+n^3+n^4$ is even in either case, $n+n^2+n^3+n^4$ is even for all integers. 
\end{p}

\begin{df}[Perfect Square]
A perfect square is an integer $n$ for which there exists an integer $m$ such that $n=m^2$. 
\end{df}

\begin{framed}\begin{q}
	{Exam 1 Review 2-f}
	{If $n$ is a positive integer such that $n$ is in the form $4k+2$ or $4k+3$, then $n$ is not a perfect square.}
\end{q}\end{framed}
\begin{p}
	We will prove the contrapositive of the statement: ``If $n$ is a perfect square, then $n$ is a positive integer of the form $4k$ or $4k+1\fs k\in\Z.$'' Suppose $n$ to be a perfect square, then $\exists m\in\Z\st n=m^2.$ $\boxed{\text{Case }1}$ Suppose $m$ is even, then $m=2t\fs t\in\Z.$\[n=m^2=(2t)^2=4t^2>0.\] Let $k=t^2$. Since $t^2\in\Z$, we have $k\in\Z.$Hence, $n$ is positive and is in the form of $4k$.\par 
	$\boxed{\text{Case }2}$ Suppose $m$ is odd, then $m=2t+1\fs t\in\Z$.\[n=m^2=(2t+1)^2=4t^2+4t+1=4(t^2+t)+1>1\] Let $k=t^2+t$. Since $(t^2+t)\in\Z$, we have $k\in\Z.$ Hence, $n$ is in the form of $4k+1$. Hence, we prove the contrapositive of the original statement to be true, which means our original statement is also true.	
\end{p}

\begin{framed}\begin{q}
	{Exam 1 Review 2-g}
	{For any integer $n$, $3\mid n$ if and only if $3\mid n^2$.}
\end{q}\end{framed}
\begin{p}
	Suppose $n\in\Z$.\par
	($\Rightarrow$) Suppose $3\mid n$. Then, $\exists k\in\Z\st n=3k.$ Then, $n^2=(3k)^2=9k^2=3(3k^2).$ Since $3k^2\in\Z$, by definition, $3\mid n^2.\qquad\square$\par
	($\Leftarrow$) WTS: $3\mid n^2\implies3\mid n$. We will prove the contrapositive: If $3\nmid n$, then $3\nmid n^2$ Suppose $3\nmid n.$ \par\hspace{5mm}
		$\boxed{\text{Case }1}$ Suppose $n=3m+1\fs m\in\Z$. Then, $n^2=(3m+1)^2=9m^2+6m+1$ Since $9m^2+6m+1$ cannot be written in the form of $3k\fs k\in\Z,$ by definition, $3\nmid n^2.$\par\hspace{5mm}
		$\boxed{\text{Case }2}$ Suppose $n=3m+2\fs m\in\Z$Then, $n^2=(3m+2)^2=9m^2+12m+4$Since $9m^2+12m+4$ cannot be written in the form of $3k$ for some $k\in\Z$, by definition, $3\nmid n^2.$ Hence, we proved the contrapositive, and thus the original statement is true.\par 
	Therefore, $n\mid n\iff 3\mid n^2.$
\end{p}

\begin{framed}\begin{q}
	{Exam 1 Review 2-h}
	{There exists an integer $n$ such that $12\mid n^2$ but $12\nmid n$.}
\end{q}\end{framed}
\begin{p}
	Observe that if we take $n=6,$ we have $n^2=36$. Since $n^2=36=3\times12,$ we know $12\mid n^2.$ However, $12\nmid6$ since $6$ cannot be written as $12k$ for all $k\in\Z.$ Hence, there exists an integer $n=6\st12\mid n^2$ but $12\nmid n$.
\end{p}

\begin{framed}\begin{q}
	{Exam 1 Review 2-i}
	{For every integer $a$, the numbers $a$ and $(a+1)(a-1)$ have opposite parity.}
\end{q}\end{framed}
\begin{p}
	Suppose $a\in\Z.$\par
	$\boxed{\text{Case }1}$ Suppose $a$ is even. Then $a=2k\fs k\in\Z.$ Then, \[(a+1)(a-1)=a^2-1=(2k)^2-1=4k^2-1=2(2k^2)-1.\] Since $2k^2\in\Z,$ we have $(a+1)(a-1)$ is odd. That is, $a$ and $(a+1)(a-1)$ have opposite parity.\par
	$\boxed{\text{Case 2}}$ Suppose $a$ is odd. Then $a=2k+1\fs k\in\Z.$ Hence, \[(a+1)(a-1)=a^2-1=(2k+1)^2-1=4k^2+4k+1-1=2(2k^2+2k).\] Since $2k^2+2k\in\Z,$ we have $(a+1)(a-1)$ is even. As a result, $a$ and $(a+1)(a-1)$ have opposite parity.\par 
	In both cases, we've shown that $a$ and $(a+1)(a-1)$ have opposite parity.
\end{p}

\begin{framed}\begin{q}
	{Exam 1 Review 2-j}
	{Suppose $x\in\R$. If $x^2$ is irrational, then $x$ is irrational.}
\end{q}\end{framed}
\begin{p}
	We will prove the contrapositive: ``If $x$ is rational, then $x^2$ is rational.'' Suppose $x\in\Q,$ then $x=\dfrac{p}{q}\fs p,q\in\Z,$ assuming $p$ and $1$ have no common factors and $q\neq0$. Then, \[x^2=\qty(\dfrac{p}{q})^2=\dfrac{p^2}{q^2}.\] As $p,q\in\Z$, we have $p^2,q^2\in\Z.$ Hence, $x^2=\dfrac{p^2}{q^2}\in\Q.$ Therefore, if $x$ is rational, so is $x^2$.	
\end{p}

\begin{framed}\begin{q}
	{Exam 1 Review 2-k}
	{For any integers $a$ and $b$, if $ab$ is even, then $a$ is even or $b$ is even.}
\end{q}\end{framed}
\begin{p}
	We will prove the contrapositive: ``If $a$ is odd and $b$ is odd, then $ab$ is odd.'' Suppose $a,b\in\Z$ and $a$ and $b$ are both odd. Then, $\exists k,l\in\Z\st a=2k+1\quad$ and $\quad b=2l+1.$ Then, \[ab=(2k+1)(2l+1)=4kl+2k+2l+1=2(2kl+k+l)+1.\] Since $2kl+k+l\in\Z,$ we have $ab$ is odd.
\end{p}

\begin{framed}\begin{q}
	{Exam 1 Review 2-l}
	{For $n\in\N$, $n$, $n+2$, and $n+4$ are all prime if and only if $n=3$.}
\end{q}\end{framed}
\begin{p}
	($\Rightarrow$) WTS: $n, n+2,$ and $n+4$ are all prime $\implies$ $n=3.$ We will prove the contrapositive: $n\neq3\implies n, n+2,$ or $n+4$ is not prime.\par 
		$\boxed{\text{Case }1}$ Suppose $0<n<3.$ 
			\begin{enumerate}
				\item[\ding{172}] If $n=1,$ then $n=1$ is not a prime.
				\item[\ding{173}] If $n=2,$ then $n=2$ is a prime number, but $n+2=2+2=4$ is not a prime.
			\end{enumerate}
			Hence, if $0<n<3,$ $n, n+2,$ or $n+4$ is not a prime.\par 
		$\boxed{\text{Case 2}}$ Suppose $n>3.$
		\begin{enumerate}
				\item[\ding{172}] If $n=3k\fs k\in\Z,$ then $n$ is not a prime because $3\mid n$.
				\item[\ding{173}] If $n=3k+1\fs k\in\Z,$ then $n+2=3k+1+2=3k+3=3(k+1).$ Since $k+1\in\Z,$ we have $3\mid n+2.$ Then, $n+2$ is not a prime.
				\item[\ding{174}] If $n=3k+2\fs k\in\Z,$ then $n+4=3k+2+4=3k+6=3(k+2).$ Since $k+2\in\Z,$ we know that $3\mid n+4.$ Therefore, $n+4$ is not a prime.
			\end{enumerate}
			Hence, if $n>3,$ we also have $n, n+2,$ or $n+4$ is not a prime.\par 
		In both cases, we have proven that if $n\neq3,$ then $n, n+2,$ or $n+4$ is not a prime. $\qquad\square$\par 
	($\Leftarrow$) Note that when $n=3,$ we have $n+2=3+2=5$ and $n+4=3+4=7.$ Since $3, 5,$ and $7$ are all primes, we have shown that when $n=3$, $n$, $n+2$, and $n+4$ are all primes.
\end{p}

\begin{framed}\begin{q}
	{Exam 1 Review 3-a}
	{Prove or disprove: Every real number is less than or equal to its square.}
\end{q}\end{framed}
\begin{dis}
	We will prove the negation: ``Some real number is greater than its square.''	Observe that when $x=0.1,$ then $x^2=(0.1)^2=0.01.$ Since $0.01<0.1,$ we have $x=0.1\in\R$ is greater than its square. Since the negation is true, the original statement is then false.
\end{dis}

\begin{framed}\begin{q}
	{Exam 1 Review 3-b}
	{Prove or disprove: The sum of two integers is never equal to their product.}
\end{q}\end{framed}
\begin{dis}
	We will prove the negation: ``The sum of some integers is equal to their product.'' Suppose $p,q\in\Z,$ and their sum equals to their product. Then, $p+q=pq.$ Divide $p$ on both sides: $q=1+\dfrac{q}{p}.$ Observe that when $p=2,$ we have $q=1+\dfrac{q}{2}.$ So, $2q=1+q,$ or $q=2$. Hence, $p+q=2+2=4$ and $pq=2\times2=4.$ Therefore, we've found integers $p=2$ and $q=2$ such that $p+q=pq.$
\end{dis}

\begin{framed}\begin{q}
	{Exam 1 Review 3-c}
	{Prove or disprove: There exists a non-zero integer whose cube equals its negative.}
\end{q}\end{framed}
\begin{dis}
	We will prove the negation: ``For all non-zero integers, their cubes do not equal their negations.'' Assume for the sake of contradiction that there exists a non-zero integer whose cube equals its negative. Suppose $x\in\Z$ and $x\neq0\st x^3=-x.$ So we have $x^3+x=0,$ or $x(x^2+1)=0.$ Then, $x=0$ or $x^2+1=0.$ As $x\neq0,$ it must be that $x^2+1=0,$ or $x^2=-1.$\begin{center}$\divideontimes$ This contradicts with the fact that $\forall x\in\Z, x^2\geq0>-1.$\end{center} So, our assumption is incorrect. For all non-zero integers, their cubes do not equal their negatives.
\end{dis}

\begin{framed}\begin{q}
	{Exam 1 Review 3-d}
	{Prove or disprove: Fall all $x\in\R$, $x\leq x^2$ or $0\leq x<1$.}
\end{q}\end{framed}
\begin{p}
	Suppose $x\in\R$.\par 
	$\boxed{\text{Case }1}$ Suppose $0\leq x<1.$ Then, $x$ satisfies the requirement.\par 
	$\boxed{\text{Case }2}$	Suppose $x<0,$ then $x^2>0.$ Therefore, $x<0<x^2.$\par 
	$\boxed{\text{Case }3}$ Suppose $x\geq1.$ Multiply the inequality by $x$ on both sides, we have: $x\cdot x\geq x$ or $x^2\geq x.$ Hence, $x\leq x^2.$\par 
	In all cases, we've proven that $\forall x\in\R, x\leq x^2$ or $0\leq x<1.$
\end{p}

\begin{framed}\begin{q}
	{Chapter 1.4 \# 20-a}
	{Let $n$ be an integer. Prove that $n$ is even if and only if $n^3$ is even.}
\end{q}\end{framed}
\begin{p}
	($\Rightarrow$) WTS: $n$ is even $\implies n^3$ is even. Suppose $n$ is even. Then $n=2k\fs k\in\Z.$ Then, $n^3=(2k)^3=8k^3=2(4k^3).$ Since $4k^3\in\Z,n^3$ is even.\par  
	($\Leftarrow$) WTS: $n^3$ is even $\implies n$ is even. We will prove the contrapositive: $n$ is odd $\implies n^3$ is odd. Suppose $n$ is odd. Then, $n=2k+1\fs k\in\Z.$  Then, \[n^3=(2k+1)^3=8k^3+12k^2+8k+1=2(4k^3+6k^2+4k)+1.\] Since $4k^3+6k^2+4k\in\Z,n^3$ is odd.
\end{p}

\begin{framed}\begin{q}
	{Chapter 1.4 \# 20-b}
	{Let $n$ be an integer. Prove that $n$ is odd if and only if $n^3$ is odd.}
\end{q}\end{framed}
\begin{p}
	($\Rightarrow$) WTS: $n$ is odd $\implies n^3$ is odd. This statement is previously proven.\par 
	($\Leftarrow$) WTS: $n^3$ is odd $\implies n$ is odd. We will prove the contrapositive: $n$ is even $\implies n^3$ is even. The contrapositive is also previously proven.
\end{p}

\begin{framed}\begin{q}
	{Chapter 1.4 \# 21}
	{Prove that $\sqrt[3]{2}$ is irrational.}
\end{q}\end{framed}
\begin{p}
	Assume for the sake of contradiction that $\sqrt[3]{2}$	is rational. Suppose $\sqrt[3]{2}$ is rational. By definition, $\exists p,q\in\Z\st\sqrt[3]{2}=\dfrac{p}{q},$ assuming $p$ and $q$ have no common factors and $q\neq0.$ Raise the two sides of the equation to cube: \[2=\qty(\dfrac{p}{q})^3=\dfrac{p^3}{q^3}.\] Then, $p^3=2q^3.$ Since $q^3\in\Z,$ we know $p^3$ is even. Then, $p$ is also even (previously proven). Then, $p=2k\fs k\in\Z.$ Hence,\[\begin{aligned}2q^3=p^3=(2k)^3&=8k^3\\q^3=4k^3&=2(2k^3)\end{aligned}\] Since $2k^3\in\Z,$ we see $q^3$ is even. Then, $q$ is also even. \begin{center}$\divideontimes$ This contradicts with our assumption that $p$ and $q$ have no common factors as $p,q$ being even indicates they have $2$ as their common factor.\end{center} So, our assumption is wrong, and $\sqrt[3]{2}$ is irrational.
\end{p}

\newpage
\section{Sets}
\begin{framed}\begin{q}
	{Handout Chapter 2.1 - Sets and Subsets}
	{Prove that $\{12a+4b\mid a,b\in\Z\}=\{4c\mid c\in\Z\}$.}
\end{q}\end{framed}
\begin{p}
	($\subseteq$) Suppose $x\in\qty{12a+4b\mid a,b\in\Z}$. Then, $x=12a+4b\fs a,b\in\Z.$ So, $x=12a+4b=4(3a+b).$ As $3a+b\in\Z$, we have $x\in\qty{4c\mid c\in\Z}$. By definition, $\qty{12a+4b\mid a,b\in\Z}\subseteq\qty{4c\mid c\in\Z}$.\par
	($\supseteq$) Suppose $x\in\qty{4c\mid c\in\Z}$. Then, $x=4c\fs c\in\Z.$ Suppose $c=3a+b\fs a,b\in\Z.$ Then, $x=4c=4(3a+b)=12a+4b.$ By definition, $\qty{4c\mid c\in\Z}\subseteq\qty{12a+4b\mid a,b\in\Z}$\par
	Hence, we have proven $\{12a+4b\mid a,b\in\Z\}=\{4c\mid c\in\Z\}.$
\end{p}

\begin{framed}\begin{q}
	{Exam 1 Review 2-m}
	{If $A=\{x\mid x=n^4-1,\ n\in\Z\}$ and $B=\{x\mid x=m^2-1,\ m\in\Z\}$, then $A\subseteq B$.}
\end{q}\end{framed}
\begin{p}
	Suppose $x\in A$. Then, $x=n^4-1\fs n\in\Z.$ Then, $x=n^4-1=(n^2)^2-1.$ Since $n^2\in\Z,$ we have $x\in B.$ Therefore, $A\subseteq B.$	
\end{p}

\begin{framed}\begin{q}
	{Exam 1 Review 2-n}
	{If $A$, $B$, and $C$ are sets, then $A\cap(B\cup C)=(A\cap B)\cup(A\cap C)$.}
\end{q}\end{framed}
\begin{p}
	($\subseteq$) Suppose $x\in A\cap(B\cup C).$ WTS: $A\cap(B\cup C)\subseteq(A\cap B)\cup(A\cap C).$ By definition, $x\in A$ and $x\in(B\cup C)$. By definition, $x\in A$ and $x\in B$ or $x\in A$ and $x\in C.$ Therefore, $x\in(A\cap B)$ or $x\in(A\cap C).$ That is, $x\in(A\cap B)\cup(A\cap C).$ Hence, $A\cap(B\cup C)\subseteq(A\cap B)\cup(A\cap C).\qquad\square$\par 
	($\supseteq$) Suppose $x\in(A\cap B)\cup(A\cap C).$ WTS: $(A\cap B)\cup(A\cap C)\subseteq A\cap(B\cup C).$ By definition, $x\in(A\cap B)$ or $x\in(A\cap C).$ WLOG, consider $x\in(A\cap B).$ Then, $x\in A$ and $x\in B.$ Similarly, we know $x\in A$ and $x\in C$ from $x\in(A\cap C).$ Therefore, $x\in A$ and $x\in B$ or $x\in C.$ That is, $x\in A$ and $x\in(B\cup C),$ or $x\in A\cap(B\cup C).$ Hence, $(A\cap B)\cup(A\cap C)\subseteq A\cap(B\cup C).$\par 
	As $A\cap(B\cup C)\subseteq(A\cap B)\cup(A\cap C)$ and $(A\cap B)\cup(A\cap C)\subseteq A\cap(B\cup C),$ we have shown that $A\cap(B\cup C)=(A\cap B)\cup(A\cap C).$
\end{p}

\begin{framed}\begin{q}
	{Exam 1 Review 2-o}
	{For subsets $A$ and $B$ of a universal set $U$, $\overline{A\cup B}=\overline{A}\cap\overline{B}$.}
\end{q}\end{framed}
\begin{p}
	($\subseteq$) Suppose $x\in\overline{A\cup B}.$ By definition, $x\notin A\cup B$. That is, $x\notin A$ and $x\notin B.$ Or, $x\in\overline{A}$ and $x\in\overline{B}.$ That is, $x\in\overline{A}\cap\overline{B}.$ Therefore, $\overline{A\cup B}\subseteq\overline{A}\cap\overline{B}.\qquad\square$\par 
	($\supseteq$) Suppose $x\in\overline{A}\cap\overline{B}.$ By definition, $x\notin A$ and $x\notin B$. That is, $x\in\overline{A\cup B}.$ Therefore, $\overline{A}\cap\overline{B}\subseteq\overline{A\cup B}.$\par 
	Since $\overline{A\cup B}\subseteq\overline{A}\cap\overline{B}$ and $\overline{A}\cap\overline{B}\subseteq\overline{A\cup B},$ we have $\overline{A\cup B}=\overline{A}\cap\overline{B}.$
\end{p}

\begin{framed}\begin{q}
	{Exam 1 Review 2-p}
	{Suppose that $A$, $B$, and $C$ are subsets of a universal set $U$. Let $P$ and $Q$ be the following statements:}\par\hspace{5mm}\texttt{$P$: $A\subseteq B$ or $A\subseteq C$; and}\par\hspace{5mm}\texttt{$Q$: $A\subseteq B\cap C$.}\par\noindent\texttt{Write the statement $P\implies Q$, its converse, and its contrapositive. Prove the true ones or give counterexamples.}
\end{q}\end{framed}
\begin{clm}
	$P\implies Q:$ $A\subseteq B$ or $A\subseteq C\implies A\subseteq B\cap C.$
\end{clm}
\begin{p}
	Suppose $x\in A.$\par 
	$\boxed{\text{Case }1}$ Suppose $A\subseteq B.$ Then $x\in B.$ Since $B\cap C\subseteq B, x\in B\cap C.$ Therefore, $A\subseteq B\cap C.$\par 
	$\boxed{\text{Case }2}$ Suppose $A\subseteq C.$ Then $x\in C.$ Since $B\cap C\subseteq C,x\in B\cap C.$ Therefore, $A\subseteq B\cap C.$\par 
	In both cases, we proven $A\subseteq B$ or $A\subseteq C\implies A\subseteq B\cap C.$
\end{p}
\begin{clm}
	Converse: $Q\implies P$: $A\subseteq B\cap C\implies A\subseteq B$ or $A\subseteq C.$
\end{clm}
\begin{p}
	Suppose $A\subseteq B\cap C.$ Suppose $x\in A.$ Then $x\in B\cap C.$ By definition, $x\in B$ and $x\in C$. Hence, $A\subseteq B$ and $A\subseteq C.$ Since the ``or'' here is inclusive, $A\subseteq B$ and $A\subseteq C$ is a true case for $A\subseteq B$ or $A\subseteq C.$ Hence, $A\subseteq B\cap C\implies A\subseteq B$ or $A\subseteq C.$
\end{p}
\begin{clm}
	Contrapositive: $\neg Q\implies\neg P$: $A\nsubseteq B\cap C\implies A\nsubseteq B$ and $A\nsubseteq C.$	
\end{clm}
\begin{p}
	Since the original statement is true, its contrapositive is automatically true.
\end{p}

\begin{framed}\begin{q}
	{Handout Chapter 2.2 \# 10-a-i}
	{Let $A=\{6a+4\mid x\in\Z\}$ and $B=\{18b-a\mid b\in\Z\}$. Prove or disprove: $A\subseteq B$.}
\end{q}\end{framed}
\begin{dis}
	Suppose $x\in A$. Then $x=6a+3\fs a\in\Z.$ Notice that $6a+4=18\qty(\dfrac{1}{3}a+\dfrac{1}{3})-2.$ Since $\dfrac{1}{3}a+\dfrac{1}{3}=\dfrac{1}{3}(a+1)\in\Q,$ but $\dfrac{1}{3}(a+1)\notin\Z\forall a\in\Z,$ we have $6a+4\notin\qty{18b-2\mid b\in\Z}.$ By definition of subsets, $A\nsubseteq B.$
	\begin{rmk}
		We can also use proof by contradiction to disprove this statement.
	\end{rmk}
\end{dis}

\begin{framed}\begin{q}
	{Handout Chapter 2.2 \# 10-a-ii}
	{Let $A=\{6a+4\mid x\in\Z\}$ and $B=\{18b-a\mid b\in\Z\}$. Prove or disprove: $B\subseteq A$.}
\end{q}\end{framed}
\begin{p}
	Suppose $x\in B.$ Then, $x=18b-2\fs b\in\Z.$ Notice that $18b-2=6(3b-1)+4.$ Since $3b-1\in\Z,$ we have $x\in A.$ Hence, by definition of subsets, $B\subseteq A.$	
\end{p}

\begin{framed}\begin{q}
	{Handout Chapter 2.2 \# 10-b}
	{If $A$ and $B$ are sets, then $\pwer(A)-\pwer(B)=\pwer(A-B)$.}
\end{q}\end{framed}
\begin{p}
	($\subseteq$) WTS: $\pwer(A)-\pwer(B)\subseteq\pwer(A-B).$ Suppose $X\in\pwer(A)-\pwer(B).$ By definition of set difference, $X\in\pwer(A)$ and $X\notin\pwer(B).$ By definition of power sets, $X\subseteq A$ and $X\nsubseteq B.$ Hence, $X\subseteq(A-B)$, by definition of set difference. Therefore, $X\in\pwer(A-B),$ and thus $\pwer(A)-\pwer(B)\subseteq\pwer(A-B)$ as desired.$\qquad\square$\par 
	($\supseteq$) WTS: $\pwer(A-B)\subseteq\pwer(A)-\pwer(B).$ Suppose $X\in\pwer(A-B).$ Then, $X\subseteq A-B.$ By definition of set difference, $X\subseteq A$ and $X\nsubseteq B.$ Then, $X\in\pwer(A)$ and $X\notin\pwer(B).$ By definition of set difference, $X\in\pwer(A)-\pwer(B).$ Hence, $\pwer(A-B)\subseteq\pwer(A)-\pwer(B).$
\end{p}

\begin{framed}\begin{q}
	{Handout Chapter 2.2 \# 10-c}
	{If $A$, $B$, and $C$ are sets, and $A\cross B=B\cross C$, then $A=B$.}
\end{q}\end{framed}
\begin{p}
	Suppose $A,B,$ and $C$ are sets. Suppose $\exists a,b\in\Z\st(a,c)\in A\times C.$ By definition of Cartesian product, $a\in A$ and $c\in C.$ Suppose $\exists b,c\in\Z\st(b,c)\in B\times C.$ So, we know that $b\in B.$ Suppose $A\times C=B\times C.$ Then, $A\times C\subseteq B\times C$ and $A\times C\supseteq B\times C.$\par 
	($\subseteq$) If $A\times C\subseteq B\times C,$ we have $(a,c)\in B\times C.$ Then, $a\in B.$ Since $a\in A,$ we know $A\subseteq B.\qquad\square$\par 
	($\supseteq$) Similarly, since $A\times C\supseteq B\times C,$ we have $(b,c)\in A\times C.$ Then, $b\in A.$ Since $b\in B,$ we see that $B\subseteq A.$\par 
	By definition of set equality, $A=B.$
\end{p}

\begin{framed}\begin{q}
	{Chapter 2.1 \# 6}
	{Let $n\in\Z$ and let $A=n\Z$. Prove that if $x,y\in A,$ then $x+y\in Z$ and $xy\in A$.}
\end{q}\end{framed}
\begin{p}
	Suppose $n\in\Z$ and $A=n\Z.$ Then, $A=\qty{nk\mid k\in\Z}.$ Suppose $x,y\in A.$ Then, $\exists k,l\st x=nk$ and $y=nl.$ Then, $x+y=nk+nl=n(k+l).$ Since $k+l\in\Z, x+y\in A.$ Similarly, $xy=(nk)(nl)=n(nkl).$ Since $nkl\in\Z,xy\in A.$
\end{p}

\begin{framed}\begin{q}
	{Chapter 2.1 \# 10}
	{Let $n$ and $m$ be integers. Let $A=n\Z$ and $B=m\Z.$ Prove that if $n$ is a multiplier of $m$, then $A\subseteq B.$}
\end{q}\end{framed}
\begin{p}
	Let $n$ and $m$ be integers. Let $A=n\Z$ and $B=m\Z.$ Suppose $x\in A$. Then, by definition, $\exists k\in\Z\st x=nk.$ Since $n$ is a multiplier of $m$, $n=ml\fs l\in\Z.$ Then, $x=nk=(ml)k=m(lk).$ Since $lk\in\Z,$ $x=m(lk)$ is a multiplier of $m$. That is, $x\in m\Z.$ Hence, $A\subseteq B.$
\end{p}

\begin{framed}\begin{q}
	{Chapter 2.1 \# 12}
	{Let $A=\qty{n\in\Z\mid n\text{ is a multiple of }4}$ and $B=\qty{n\in\Z\mid n^2\text{ is a multiple of }4}$. Prove that $A\subseteq B$ and $B\nsubseteq A$.}
\end{q}\end{framed}
\begin{p}
	WTS: $A\subseteq B.$ Suppose $x\in A$. Then, $\exists k\in\Z\st x=4k.$ Consider $x^2=(4k)^2=16k^2=4(8k^2).$ Since $8k^2\in\Z,$ by definition of divides, $x^2$ is a multiple of $4$. Hence, by definition of set $B$, $x\in B.$ That is, $A\subseteq B.$	
\end{p}

\begin{p}
	WTS: $B\nsubseteq A.$ Consider $x=2k\fs k\in\Z.$ Then, $x^2=(2k)^2=4k^2.$ Since $k^2\in\Z,$ $x^2$ is a multiple of $4$. Hence, $x\in B.$ However, $x=2k$ is not a multiple of $4$. That is, $x\notin A.$ Hence, we found an element of $B$ that is not an element of $A$. Then, by definition, $B\nsubseteq A.$
\end{p}

\begin{framed}\begin{q}
	{Chapter 2.1 \# 13}
	{If $A=\qty{n\in\Z\mid n+3\text{ is odd}}$, then $A$ is equal to the set of all even integers.}
\end{q}\end{framed}
\begin{p}
	Suppose $B=\qty{n\in\Z\mid n\text{ is even}}.$ Then, $B$ is the set of all even numbers. \par 
	($\subseteq$) Suppose $x\in A.$ Then, by definition, $x+3$ is odd. That is, $\exists k\in\Z\st x+3=2k+1.$ Then, $x=2k+1-3=2k-2=2(k-1).$ Since $k-1\in\Z,$ then $x$ is even. Therefore, $x\in B,$ and $A\subseteq B.\qquad\square$\par 
	($\supseteq$) Suppose $x\in B.$ Then, $x$ is even. So, $\exists k\in\Z\st x=2k.$ Consider $x+3=2k+3=2k+2=1=2(k+1)+1.$ Since $k+1\in\Z,$ then $x+3$ is odd. Hence, $x\in A,$ and $B\subseteq A.$\par 
	Collectively, we've proven $A=B.$
\end{p}

\begin{framed}\begin{q}
	{Chapter 2.1 \# 15}
	{Let $A=\qty{n\in\Z\mid n=4t+1\text{ for some }t\in\Z}$ and $B=\qty{n\in\Z\mid n=4t+9\text{ for some }t\in\Z}$. Prove that $A=B.$}
\end{q}\end{framed}
\begin{p}
	($\subseteq$) Suppose $x\in A.$ Then, $x=4t+1\fs t\in\Z.$ Note that $x=4t+9-8=(4t-8)+9=4(t-2)+9.$ Since $t-2\in\Z,$ by definition, $x\in B.$ Then, $A\subseteq B.\qquad\square$\par 
	($\supseteq$) Suppose $x\in B.$ Then, $x=4t+9\fs t\in\Z.$ Note that $x=4t+9=4t+8+1=4(t+2)+1.$ Since $t+2\in\Z,$ by definition, $x\in A.$ Hence, $B\subseteq A.$\par 
	Collectively, we've proven $A=B.$
\end{p}

\begin{framed}\begin{q}
	{Chapter 2.1 \# 16}
	{Let $A=\qty{n\in\Z\mid n=3t+1\text{ for some }t\in\Z}$ and $B=\qty{n\in\Z\mid n=3t+2\text{ for some }t\in\Z}$. Prove that $A$ and $B$ have no elements in common.}
\end{q}\end{framed}
\begin{p}
	Assume for the sake of contradiction that $A$ and $B$ have one element in common, and suppose that element is $x$. By our assumption, $x\in A.$ So, $x=3t+1\fs t\in\Z.$ Also, $x\in B,$ so $x=3s+2\fs s\in\Z.$ Then, we have $x=3t+1=3s+2.$ Solve for $t$, we have \[\begin{aligned}3t&=3s+2-1=3s+1\\t&=\dfrac{3s+1}{3}=s+\dfrac{1}{3}\end{aligned}\] Since $s\in\Z,\dfrac{1}{3}\notin\Z,$ we have $t=s+\dfrac{1}{3}\notin\Z.$ \begin{center}$\divideontimes$ This contradicts with the fact that $t\in\Z.$\end{center} So, our assumption is wrong, and $A$ and $B$ have no elements in common.
\end{p}

\begin{framed}\begin{q}
	{Chapter 2.3 \# 8}
	{Let $A_i=(-i,i)=\qty{x\in\R\mid-i<x<i}$. Prove that $\dsst\bigcup_{i=1}^\infty(-i,i)=\R$ and $\dsst\bigcap_{i=1}^\infty(-i,i)=(-1,1).$}
\end{q}\end{framed}
\begin{p}
	WTS: $\dsst\bigcup_{i=1}^\infty(-i,i)=\R$\par 
	($\subseteq$) Suppose for some $k\in\Z$ and $k\geq1,$ $x\in A_k.$ That is, $x\in(-k,k).$ Since $k\geq1,$ by definition of union, $A_k\subseteq\dsst\bigcup_{i=1}^\infty(-i,i).$ Hence, $x\in\dsst\bigcup_{i=1}^\infty(-i,i).$ Since $A_k\subseteq\R, x\in\R.$ Hence, $\dsst\bigcup_{i=1}^\infty(-i,i)\subseteq\R.\qquad\square.$\par 
	($\supseteq$) Suppose $x\in\R.$ Consider the set $(-k,k)=A_k,$ where $k\in\Z$ and $k\geq x.$ Then, $x\in(-k,k).$ Since $k\in\Z,$ then $A_k\subseteq\dsst\bigcup_{i=1}^\infty(-i,i)$ by definition of union. Then, $x\in\dsst\bigcup_{i=1}^\infty(-i,i).$ That is, $\R\subseteq\dsst\bigcup_{i=1}^\infty(-i,i).$\par 
\end{p}
\begin{p}
	WTS: $\dsst\bigcap_{i=1}^\infty(-i,i)=(-1,1).$\par 
	($\subseteq$) Let $x\in\dsst\bigcap_{i=1}^\infty(-i,i).$ So, $x\in A_i\quad\forall i=\qty{1,2,3,\cdots}$. Specially, $x\in A_1=(-1,1).$ Hence, $\dsst\bigcap_{i=1}^\infty(-i,i)\subseteq(-1,1).\qquad\square$\par 
	($\supseteq$) Let $x\in(-1,1).$ Let $k\in\qty{1,2,3,\cdots}.$ We will show $x\in A_k.$ Since $k\geq1,$ then $-k\leq-1.$ Form $x\in(-1,1),$ we know $-1<x<1.$ Then, $-k\leq-1<x<1\leq k.$ That is, $-k<x<k,$ or $x\in(-k,k)=A_k.$ Since $k$ is arbitrary, we've proven $x\in A_k\quad\forall k\geq1.$ So, $x\in\dsst\bigcap_{i=1}^\infty(-i,i).$ Hence, $(-1,1)\subseteq\dsst\bigcap_{i=1}^\infty(-i,i).$
\end{p}

\begin{framed}\begin{q}
	{Chapter 2.3 \# 10}
	{Let $A_i=\qty{1,2,3,\cdots,i}$ for $i\in\Zp$. Compute $\dsst\bigcup_{i=1}^\infty A_i$ and $\dsst\bigcap_{i=1}^\infty A_i.$ Prove your answer.}
\end{q}\end{framed}
\begin{clm}
	$\dsst\bigcup_{i=1}^\infty A_i=\Zp.$
\end{clm}
\begin{p}
	($\subseteq$) Let $x\in\dsst\bigcup_{i=1}^\infty A_i.$ Then $x\in A_k\fs k\in\Zp.$ That is, by definition, $x\in\qty{1,2,3,\cdots,k}.$ Since $k\in\Zp, \qty{1,2,3,\cdots,k}\subseteq\Zp,$ $x\in\Zp.\qquad\square$\par 
	($\supseteq$)	 Let $x\in\Zp.$ Consider $A_{x+1}=\qty{1,2,3,\cdots,x+1}.$ Then, $x\in A_{x+1}.$ By definition of union, $A_{x+1}\subseteq\dsst\bigcup_{i=1}^\infty A_i.$ So, $x\in\dsst\bigcup_{i=1}^\infty A_i.$\par 
	Hence, we've shown $\dsst\bigcup_{i=1}^\infty A_i=\Zp.$
\end{p}
\begin{clm}
	$\dsst\bigcap_{i=1}^\infty A_i=\qty{1}.$
\end{clm}
\begin{p}
	($\subseteq$) Suppose $x\in\dsst\bigcap_{i=1}^\infty A_i.$ By definition of union, $x\in A_k\quad\forall k\geq1.$ Specially, $x\in A_1=\qty{1}.\qquad\square$\par 
	($\supseteq$) Suppose $x\in\qty{1}.$ Let $k\geq1.$ By definition, $A_k=\qty{1,2,3,\cdots.k}.$ Since $\qty{1}\subseteq\qty{1,2,3,\cdots,k}=A_k, x\in A_k.$ As $k$ was arbitrary, we've proven $x\in A_k\quad\forall k\geq1.$ So, $x\in\dsst\bigcap_{i=1}^\infty A_i.$ Hence, $\qty{1}\subseteq\dsst\bigcap_{i=1}^\infty A_i.$
\end{p}

\begin{framed}\begin{q}
	{Chapter 2.3 \# 10}
	{Let $A_i=[i,i+1)=\qty{x\in\R\mid i\leq x<i+1}$ for $i\in\Zp$. Compute $\dsst\bigcup_{i=1}^\infty A_i$ and $\dsst\bigcap_{i=1}^\infty A_i.$ Prove your answer.}
\end{q}\end{framed}
\begin{clm}
	$\dsst\bigcup_{i=1}^\infty A_i=\qty{x\in\R\mid x\geq1}.$
\end{clm}
\begin{p}
	($\subseteq$) Suppose $x\in\dsst\bigcup_{i=1}^\infty A_i.$ By definition of union, $x\in A_k\fs k\in\qty{1,2,\cdots}.$ By definition, $A_k=[k,k+1),$ so $k\leq x<k+1.$ Since $k\geq1,$ we have $1\leq k\leq x<k+1.$ That is, $x\in\qty{x\in\R\mid x\geq1}.$ Hence, $\dsst\bigcup_{i=1}^\infty A_i\subseteq\qty{x\in\R\mid x\geq1}.\qquad\square$\par 
	($\supseteq$) Suppose $x\in\qty{x\in\R\mid x\geq1}.$ Then, $x\geq1.$ Consider $A_x=[x,x+1),$ we have $x\in[x,x+1).$ By definition of union, $A_x\subseteq\dsst\bigcup_{i=1}^\infty A_i.$ Hence, $x\in\dsst\bigcup_{i=1}^\infty A_i,$ or $\qty{x\in\R\mid x\geq1}\subseteq\dsst\bigcup_{i=1}^\infty A_i.$
\end{p}
\begin{clm}
	$\dsst\bigcap_{i=1}^\infty A_i=\emptyset.$
\end{clm}
\begin{p}
	Note that $n+1\in A_{n+1}$.	However, $n+1\notin A_n=[n,n+1).$ That is, for every $n\in\Zp,$ $n+1$ is not in every $A_i.$ So, by definition of set intersection, $\dsst\bigcap_{i=1}^\infty A_i=\emptyset.$
\end{p}

\begin{framed}\begin{q}
	{Chapter 2.3 \# 12}
	{Let $A_i=\left(\dfrac{1}{i},i\right]=\qty{x\in\R\mid \dfrac{1}{i}<x\leq i}$ for $i\geq2$. Compute $\dsst\bigcup_{i=1}^\infty A_i$ and $\dsst\bigcap_{i=1}^\infty A_i.$ Prove your answer.}
\end{q}\end{framed}
\begin{clm}
	$\dsst\bigcup_{i=1}^\infty A_i=(0,\infty).$
\end{clm}
\begin{p}
	Prove.	
\end{p}
\begin{clm}
	$\dsst\bigcap_{i=1}^\infty A_i=\left(\dfrac{1}{2},2\right].$
\end{clm}
\begin{p}
	Prove.	
\end{p}

\begin{framed}\begin{q}
	{Chapter 2.3 \# 13}
	{Let $A_i=\qty[i,1+\dfrac{1}{i}]$ for $i\in\Zp$. Compute $\dsst\bigcup_{i=1}^\infty A_i$ and $\dsst\bigcap_{i=1}^\infty A_i.$ Prove your answer.}
\end{q}\end{framed}
\begin{clm}
	$\dsst\bigcup_{i=1}^\infty A_i=[1,2].$
\end{clm}
\begin{p}
	Prove.	
\end{p}
\begin{clm}
	$\dsst\bigcap_{i=1}^\infty A_i=\qty{1}.$
\end{clm}
\begin{p}
	Prove.	
\end{p}

\begin{framed}\begin{q}
	{Chapter 2.3 \# 14}
	{Let $A_i=\qty(i,1+\dfrac{1}{i})$ for $i\in\Zp$. Compute $\dsst\bigcup_{i=1}^\infty A_i$ and $\dsst\bigcap_{i=1}^\infty A_i.$ Prove your answer.}
\end{q}\end{framed}
\begin{clm}
	$\dsst\bigcup_{i=1}^\infty A_i=(1,2),$ and $\dsst\bigcap_{i=1}^\infty A_i=\emptyset.$
\end{clm}
\begin{p}
	Similar proofs as done in the previous exercise.	
\end{p}

\begin{framed}\begin{q}
	{Exam 2 Review 2}
	{For sets $A,B,C,D$, prove that $(A\times B)\cap(C\times D)=(A\cap C)\times(B\cap D).$}
\end{q}\end{framed}
\begin{p}
	Prove.	
\end{p}

\begin{framed}\begin{q}
	{Exam 2 Review 3}
	{Given the indexed sets, compute the unions and intersections. Give full and careful proofs of each: $A_i=[i-1,i]$ for $i=1,\cdots,n.$ Compute $\dsst\bigcup_{i=1}^nA_i$ and $\dsst\bigcap_{i=1}^nA_i.$}
\end{q}\end{framed}
\begin{clm}
	$\dsst\bigcap_{i=1}^nA_i=\begin{cases}A_1,&n=1\\A_1\cap A_2=\qty{1},&n=2\\\emptyset,&n\geq3\end{cases}.$	
\end{clm}
\begin{p}
	Prove.	
\end{p}
\begin{clm}
	$\dsst\bigcup_{i=1}^nA_i=[0,n].$	
\end{clm}
\begin{p}
	
\end{p}

\begin{framed}\begin{q}
	{Exam 2 Review 4}
	{Here's a mathematical statement: \begin{center}($s$): for all sets $A$ and $B$, $A\subseteq B$ implies that $\pwer(A)\subseteq\pwer(B).$\end{center} State the converse ($s_1$) of ($s$), the contrapositive $(s_2)$ of ($s$), the negation ($\neg s$) of ($s$). Which of the statements ($s$), ($s_1$), ($s_2$), ($\neg s$) are true? }
\end{q}\end{framed}
\begin{clm}
	($s$) is true.	
\end{clm}
\begin{p}
	Prove.	
\end{p}
\begin{clm}
	($s_1$) is true.	
\end{clm}
\begin{p}
	Prove.	
\end{p}
\begin{clm}
	($s_2$) is true.	
\end{clm}
\begin{p}
	Prove.	
\end{p}
\begin{clm}
	($\neg s$) is false.	
\end{clm}
\begin{p}
	Prove.	
\end{p}

\begin{framed}\begin{q}
	{Exam 2 Review 5}
	{For all sets $A$ and $B$, if $\pwer(A)=\pwer(B),$ then $A=B.$}
\end{q}\end{framed}
\begin{p}
	Prove.	
\end{p}

\begin{framed}\begin{q}
	{Exam 2 Review 7}
	{Find $\dsst\bigcap_{n\in\N}=n\Z$}
\end{q}\end{framed}
\begin{clm}
	
\end{clm}
\begin{p}
	Prove.	
\end{p}

\newpage
\section{Integers and Induction}
\begin{framed}\begin{q}
	{Handout Chapter 5.1-5.2-Axioms of Integers}
	{Let $a,b\in\Z.$ Then $(-a)(-b)=ab$.}
\end{q}\end{framed}
\begin{p}
	Notice that $a\cdot0=0$. Multiply $(-1)$ on both sides: \[\begin{aligned}(-a\cdot0)&=-0=0\\(-a)\cdot0&=0\end{aligned}\] By additive identity, $b+(-b)=0,$ so we know that \[(-a)(b+(-b))=0.\] By distributivity, \[(-a)b+(-a)(-b)=0.\] Add the additive inverse of $-ab$ to both sides: \[\begin{aligned}-ab+(-(-ab))+(-a)(-b)&=0+(-(-ab))\\0+(-a)(-b)&=0+ab\\(-a)(-b)&=ab.\end{aligned}\]
\end{p}

\begin{framed}\begin{q}
	{Chapter 5.1 \# 1-a}
	{$-(-a)=a$ for all $a\in\Z$.}
\end{q}\end{framed}
\begin{p}
	Prove.	
\end{p}

\begin{framed}\begin{q}
	{Chapter 5.1 \# 1-c}
	{$a(b-c)=ab-ac$ for all $a,b,c\in\Z$.}
\end{q}\end{framed}
\begin{p}
	Prove.	
\end{p}

\begin{framed}\begin{q}
	{Chapter 5.1 \# 2}
	{Let $a,b\in\Z$. Prove that $-(a+b)=-a-b.$}
\end{q}\end{framed}
\begin{p}
	Prove.	
\end{p}

\begin{framed}\begin{q}
	{Chapter 5.1 \# 3}
	{Let $a,b\in\Z$. Suppose that $a<b.$ Prove that $(-a)>(-b).$}
\end{q}\end{framed}
\begin{p}
	Prove.	
\end{p}

\begin{thm}[Well Ordering Principle for $\N$.]
	If $X\subseteq\N$ and $X\neq\emptyset,$ then $\exists x_0\in X\st\forall a\in X$ and $a\neq x_0,$ we have $a-x_0\in\Zp.$
\end{thm}

\begin{framed}\begin{q}
	{Exam 2 Review 6-a}
	{Every non-empty subset of the rational numbers $\Q$ contains a minimum element.}
\end{q}\end{framed}
\begin{counter}
	Prove.
\end{counter}
\begin{counter}
	Prove	
\end{counter}
\begin{p}
	Prove.
\end{p}

\begin{framed}\begin{q}
	{Exam 2 Review 8}
	{Prove that for all $n\in\N,$ \[1\cdot2+2\cdot3+3\cdot4+\cdots+n\cdot(n+1)=\dfrac{n(n+1)(n+2)}{3}.\]}
\end{q}\end{framed}
\begin{p}
	Prove.	
\end{p}

\begin{df}[Fibonacci Sequence]
The Fibonacci Sequence $f_n$ is defined recursively as follows: \[f_1=1,\quad f_2=1,\quad\text{and}\quad f_n=f_{n-1}+f_{n-2}\text{ for }n\geq3.\]	
\end{df}


\begin{framed}\begin{q}
	{Exam 2 Review 9}
	{Prove that for all $n\in\N,$ \[f_{n+1}^2-f_{n+1}f_n-f_n^2=(-1)^n.\]}
\end{q}\end{framed}
\begin{p}
	Prove.	
\end{p}

\begin{framed}\begin{q}
	{Exam 2 Review 10}
	{Let $f:\N\to\N$ be defined recursively by $f(1)=1$ and $f(n+1)=\sqrt{2+f(n)}$ for all $n\in\N.$ Prove that $f(n)<2$ for all $n\in\N.$}
\end{q}\end{framed}
\begin{p}
	Prove.	
\end{p}

\begin{framed}\begin{q}
	{Exam 2 Review 11}
	{Prove that $1^3+2^3+3^3+\cdots+n^3=\dfrac{n^2(n+1)^2}{4}.$}
\end{q}\end{framed}
\begin{p}
	Prove.	
\end{p}

\begin{framed}\begin{q}
	{Exam 2 Review 18}
	{Let $n\in\Z$ and let $S\subseteq\Z$ satisfy $\qty|S|>n.$ Then, at least two distinct members of $S$ are congruent $\mod n$.}
\end{q}\end{framed}
\begin{p}
	Prove.	
\end{p}

\newpage
\section{Equivalence Relations}
\begin{framed}\begin{q}
	{Exam 2 Review 6-b}
	{Suppose that $R$ is an equivalence relation on $A$ and that $a,b\in A.$ Then, if $[a]\cap[b]\neq\emptyset,$ then $[a]=[b].$}
\end{q}\end{framed}
\begin{p}
	Prove.	
\end{p}

\begin{framed}\begin{q}
	{Exam 2 Review 12-a}
	{Determine whether each of the following relations on $\R$ is an equivalence relation. Justify your answer. If $R$ is an equivalence relation, describe its equivalence classes: $xRy$ if $x-y\in\Z$.}
\end{q}\end{framed}
\begin{p}
	Prove.	
\end{p}
\begin{clm}
	
\end{clm}
\begin{p}
	
\end{p}

\begin{framed}\begin{q}
	{Exam 2 Review 12-b}
	{Determine whether each of the following relations on $\R$ is an equivalence relation. Justify your answer. If $R$ is an equivalence relation, describe its equivalence classes: $xRy$ if $x+y\in\Z$.}
\end{q}\end{framed}
\begin{dis}
	Prove.	
\end{dis}

\begin{framed}\begin{q}
	{Exam 2 Review 13}
	{Prove or disprove: $R$ is an equivalence relation on $\Z$. If $R$ is an equivalence relation, describe its equivalence classes: $xRy$ if $4\mid(x+y)$. }
\end{q}\end{framed}
\begin{dis}
	Prove.	
\end{dis}

\begin{framed}\begin{q}
	{Exam 2 Review 14}
	{Prove or disprove: $R$ is an equivalence relation on $\Z$. If $R$ is an equivalence relation, describe its equivalence classes: $xRy$ if $4\mid(x+3y)$. }
\end{q}\end{framed}
\begin{p}
	Prove.	
\end{p}
\begin{clm}
	
\end{clm}
\begin{p}
	
\end{p}

\begin{framed}\begin{q}
	{Exam 2 Review 15}
	{Define a relation $R$ on $\R^2$ as follows: for all $(a_1,b_1),(a_2,b_2)\in\R^2,(a_1,b_1)R(a_2,b_2)$ if $(a_1,b_1)$ and $(a_2,b_2)$ are on the same line through the origin. Decide whether $R$ is an equivalence relation - either show why or why not. If it is, what are the elements of the equivalence class $[(1,2)]$?}
\end{q}\end{framed}
\begin{p}
	Prove.	
\end{p}
\begin{clm}
	
\end{clm}
\begin{p}
	
\end{p}

\newpage
\section{Functions}
\begin{framed}\begin{q}
	{Exam 2 Review 17}
	{Let $A=\qty{x,y,z}.$ Define functions $f:\pwer(A)$ by $f(a)=\qty{a}$ and $g:A\to\pwer(A)$ by $g(a)=A-\qty{a}.$ Find $\Im(f)$ and $\Im(g).$}
\end{q}\end{framed}
\begin{clm}
	
\end{clm}
\begin{p}
	Prove.	
\end{p}
\begin{clm}
	
\end{clm}
\begin{p}
	
\end{p}

\begin{framed}\begin{q}
	{Exam 2 Review 17}
	{Let $f:\R\to\R$ be given by $f(x)=2x^3+3x^2-12x+1.$ Let $X=\qty[-1,2].$ Find $f(X).$}
\end{q}\end{framed}
\begin{ans}
	Prove.	
\end{ans}

\begin{df}[$\epsilon-\delta$ Definition of Continuity]
	Suppose $f:\R\to\R$ is defined by $f(x)$, then $f$ is continuous at $x=a$ when then following condition is satisfied: \[\forall\epsilon>0,\exists\delta\in\R\st\qty|x-a|<\delta\implies\qty|f(x)-f(a)|<\epsilon\]
\end{df}

\begin{framed}\begin{q}
	{Exam 3 Review 2}
	{Consider the function $f(x)=\begin{cases}0,&x<0\\1,&x\geq0\end{cases}.$ Rigorously prove that $f$ is discontinuous at $x=0.$ Your proof should involve $\epsilon$ and $\delta.$}
\end{q}\end{framed}
\begin{p}
	Prove.	
\end{p}

\begin{framed}\begin{q}
	{Exam 3 Review 3-a}
	{Use the formal definition of continuity, prove that the function $f(x)=x^2+4x+3$ is continuous at $x=-2.$}
\end{q}\end{framed}
\begin{p}
	Prove.	
\end{p}

\begin{framed}\begin{q}
	{Exam 3 Review 3-b}
	{Use the formal definition of continuity, prove that the function $f(x)=x^2+4x+3$ is continuous at $x=2.$}
\end{q}\end{framed}
\begin{p}
	Prove.	
\end{p}

\begin{framed}\begin{q}
	{Exam 3 Review 6}
	{Prove or disprove: Every injective map form $\R\to\R$ is bijective.}
\end{q}\end{framed}
\begin{dis}
	Prove.	
\end{dis}

\begin{framed}\begin{q}
	{Exam 3 Review 7}
	{Show that the function $f:\R-\qty{0}\to\R$ defined by $f(x)=\dfrac{x+1}{x}$ is injective but not surjective. How could we change the codomain so that $f$ is surjective?}
\end{q}\end{framed}
\begin{p}
	Prove.	
\end{p}
\begin{ans}
	
\end{ans}

\begin{framed}\begin{q}
	{Exam 3 Review 11-a}
	{Let $f:A\to B$ for a function and $X\subseteq A$. Prove or disprove: $\f(f(X))=X.$}
\end{q}\end{framed}
\begin{dis}
	Prove.	
\end{dis}

\begin{framed}\begin{q}
	{Exam 3 Review 11-b}
	{Let $f:A\to B$ for a function and $X\subseteq A$. Prove or disprove: $f(\f(f(X)))=f(X).$}
\end{q}\end{framed}
\begin{p}
	Prove.	
\end{p}

\begin{framed}\begin{q}
	{Exam 3 Review 12}
	{Let $f:A\to B$ and $g:B\to C$, and assume that $f$ is surjective. Prove that $g\of f$ is injective if and only if $g$ and $f$ are both injective.}
\end{q}\end{framed}
\begin{p}
	Prove.	
\end{p}

\begin{framed}\begin{q}
	{Exam 3 Review 13}
	{Suppose that $f:A\to B$ is a function. Prove that $f$ is injective if and only if for all subsets $C,D$ of $A$, $f(C\cap D)=f(C)\cap f(D).$}
\end{q}\end{framed}
\begin{p}
	Prove.	
\end{p}

\begin{thm}
A function is a bijection if and only if it is invertible.	
\end{thm}

\begin{framed}\begin{q}
	{Exam 3 Review 14}
	{Let $A,B$ be sets, and let $F(A,B)$ denote the set of all functions from $A$ to $B$. Let $g:A\to A$ be a bijection. Define a new function $\Delta_g: F(A,B)\to F(A,B)$ as follows: $f\mapsto f\of g.$ Prove that $\Delta_g$ is a bijection.}
\end{q}\end{framed}
\begin{p}
	Prove.	
\end{p}

\begin{framed}\begin{q}
	{Exam 3 Review 15-a}
	{Let $A,B$ be sets, and let $f:A\to B$ be a function. Let $I$ be an index, and let $\qty{C_i}_{i\in I}$ be a collection of subsets such that for all $i\in I,C_i\subseteq B.$ Prove that $\dsst\f\qty(\bigcap_{i\in I}C_i)=\bigcap_{i\in I}\f(C_i).$}
\end{q}\end{framed}
\begin{p}
	Prove.	
\end{p}

\begin{framed}\begin{q}
	{Exam 3 Review 15-a}
	{Let $A,B$ be sets, and let $f:A\to B$ be a function. Let $I$ be an index, and let $\qty{C_i}_{i\in I}$ be a collection of subsets such that for all $i\in I,C_i\subseteq B.$ Prove that $\dsst\f\qty(\bigcup_{i\in I}C_i)=\bigcup_{i\in I}\f(C_i).$}
\end{q}\end{framed}
\begin{p}
	Prove.	
\end{p}

\label{LastPage}
\end{document}
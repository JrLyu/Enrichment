\documentclass[10pt,letter]{article}
\usepackage{amssymb,amsmath,amsthm,amsfonts,epsfig,graphicx,dsfont,bbm, bbold, url, color, setspace, multirow,pinlabel, pifont, framed, physics}
\usepackage[all]{xy}
\usepackage{fancyhdr, geometry}

% Page Setting
\geometry{left=2cm, right=2cm, bottom=2cm, top=2cm}
\pagestyle{fancy}
\linespread{1.75}

% Head & Foot Setting
\fancyhead[LO,LE]{} 
\fancyhead[RO,RE]{}
\chead{\textbf{Practice Proofs}} 
\cfoot{\textbf{Page \thepage\ of \pageref{LastPage}}}
\fancyfoot[LO,LE]{} 
\fancyfoot[RO,RE]{} 
\renewcommand{\headrulewidth}{0.5pt}
%\renewcommand{\footrulewidth}{0.5pt}
\parindent 2.5em

% Defining Environments
\newcounter{nq}[section]
\setcounter{nq}{0}
\newcounter{np}[section]
\setcounter{np}{0}
\newtheorem{thm}{Theorem}[section]
\newtheorem{df}{Definition}[section]
\newtheorem{eg}{Example}
\newtheorem{ax}{Axiom}[section]
\newtheorem{prop}{Proposition}[section]
\newtheorem{lem}{Lemma}[section]
\newtheorem{cor}{corollary}[section]
\newtheorem*{rmk}{\indent Remark}
\newtheorem*{ext}{\indent Extension}
\newenvironment*{p}{\par\noindent\textbf{\textit{Proof \stepcounter{np}\thenp. }}\par}{\par\hfill $\blacksquare$\par}
\newenvironment*{dis}{\par\noindent\textbf{\textit{Disproof \stepcounter{np}\thenp. }}\par}{\par\hfill $\blacksquare$\par}
\newenvironment*{q}[1]{\noindent\emph{\thesection.\stepcounter{nq}\thenq$\quad $ #1}\par\noindent\texttt}{}


% Defining Symbols
\def\Z{{\mathbb{Z}}}
\def\R{{\mathbb{R}}}
\def\C{{\mathbb{C}}}
\def\Q{{\mathbb{Q}}}
\def\N{{\mathbb{N}}}
\def\E{{\mathbb{E}}}
\def\emptyset{\varnothing}
\def\pwer{{\mathcal{P}}}
\def\qed{\rightline{$\blacksquare$}}
\def\st{\emph{ s.t. }}
\def\fs{\emph{ f.s. }}
\def\Iff{\emph{ iff }}

% Title Information
\title{Foundations of Mathematics -- \textbf{Proof Practice}}
\author{Jiuru Lyu}
\date{\today}

\begin{document}

\section{Statements}
\begin{framed}\begin{q}
	{Class Handout, Chapter 1.3, Implications.}
	{Let $a,\ b,\ $and $c$ be integers, with $a$ and $b$ non-zero. If $(ab)\mid(ac)$, then $b\mid c.$}
\end{q}\end{framed}
\begin{p}
	Let	$a,b,c\in\Z$ with $a\neq0$ and $b\neq0$.\par Suppose $(ab)\mid(ac)$. Then $\exists k\in\Z\st ac=(ab)k$.\par Divide both sides of the equation by $a$: \[c=bk.\] \par Since $k\in\Z$, by definition of divides, $b\mid c$.
\end{p}

\begin{framed}\begin{q}
	{Class Handout, Chapter 1.4, Contrapositive and Converse}
	{Prove that for all real numbers $a$ and $b$, if $a\in\Q$ and $ab\notin\Q$, then $n\notin Q$.}	
\end{q}\end{framed}
\begin{p}
	Let $a,b\in\Q$.\par Assume for the sake of contradiction that if $a\in\Q$ and $ab\notin\Q$, we have $b\in\Q$.\par Then, $\exists p,q,m,n\in\Z\st a=\dfrac{m}{n}$ and $b=\dfrac{p}{q}$.\par Hence, \[ab=\frac{m}{n}\cdot\frac{p}{q}=\frac{mp}{nq}\]\par As $mp, nq\in\Z$, $ab\in\Q.$\begin{center}$\divideontimes$ This contradicts with the fact that $ab\notin\Q$.\end{center}\par So, $b$ must not be rational. 
\end{p}

\begin{framed}\begin{q}
{Chapter 1.1 \# 7(c)}
{Prove the square of an even integer is divisible by 4.}	
\end{q}\end{framed}
\begin{p}
	Suppose $x\in\Z$ is even. Then $\exists k\in\Z\st x=2k.$\par Then, $x^2=(2k)^2=4k^2.$\par Since $k^2\in\Z$, we have $4\mid 4k^2$.
\end{p}

\begin{thm}[Archimedean Principle]\label{AP}
	For every real number $x$, there is an integer $n$, such that $n>x$.	
\end{thm}

\begin{framed}\begin{q}
	{Chapter 1.1 \# 11}
	{For every positive real number $\varepsilon$, there exists a positive integer $N$ such that $\dfrac{1}{n}<\varepsilon$ for all $n\geq N.$}
\end{q}\end{framed}
\begin{p}
	Suppose $\varepsilon\in\R$ and $\varepsilon>0.$\marginpar{}\par Since $\varepsilon\in\R$, we have $\dfrac{1}{n}\in\R.$\par Then, by Archimedean Principle, $\exists n\in\Z\st n>\dfrac{1}{\varepsilon}.$\par Hence, $n\varepsilon>1$ or $\varepsilon>\dfrac{1}{n}.$\par Suppose $N\in\Z\st N=\left\lceil\dfrac{1}{\varepsilon}\right\rceil$\par\textit{$\left\lceil\dfrac{1}{\varepsilon}\right\rceil$ means the integer greater to $\dfrac{1}{\varepsilon}$ if $\dfrac{1}{\varepsilon}\notin\Z$, and the integer equals to $\dfrac{1}{\varepsilon}$ if $\dfrac{1}{\varepsilon}\in\Z.$}\par Hence, $N\geq\dfrac{1}{\varepsilon}.$\par As $n>\dfrac{1}{\varepsilon},$ we have $n\geq N$
\end{p}

\begin{framed}\begin{q}
	{Chapter 1.1 \# 12}
	{Use the Archimedean Principle (Theorem \ref{AP}) to prove if $x$ is a real number, then there exists a positive integer $n$ such that $-n<x<n.$}
\end{q}\end{framed}
\begin{p}
	Suppose $x\in\R.$\par $\boxed{\text{Case }1}$ If $x>0$, then $-x<0$ (i.e., $-x<0<x$).\par\hspace{10mm} By the Archimedean Principle, $\exists n\in\Z\st n>x.$\par\hspace{10mm} Multiply $(-1)$ on both sides of the inequality: \[-n<-x\]\par\hspace{10mm} As $-x<0<x$, \[-n<-x<0<x<n,\]\par\hspace{10mm} which means $-n<x<n,$ and $n$ is positive.\par $\boxed{\text{Case }2}$ If $x<0,$ then $-x>0$ (i.e., $-x>0>x$)\par\hspace{10mm} Since $x\in\R$, we have $-x\in\R$.\par\hspace{10mm} By the Archimedean Principle, $\exists n\in\Z\st n>-x.$\par\hspace{10mm} Multiply $(-1)$ on both sides of the inequality: \[-n<x\]\par\hspace{10mm} As $x<0<-x,$ \[-n<x<0<-x<n,\]\par\hspace{10mm} which means $-n<x<n,$ and $n$ is positive.\par In all cases, we have proven that $x\in\R\implies\exists n\in\Z,\ n>0\st-n<x<n.$
\end{p}

\begin{framed}\begin{q}
	{Chapter 1.1 \# 13}
	{Prove that if $x$ is a positive real number, then there exists a positive integer $n$ such that $\dfrac{1}{n}<x<n$.}
\end{q}\end{framed}
\begin{p}
	Suppose $x\in\R,\ x>0$\par $\boxed{\text{Case }1}$ If $0<x\leq1,$ then $\dfrac{1}{x}\geq1.$\par\hspace{10mm} Hence, $x\leq1\leq\dfrac{1}{x}.$\par\hspace{10mm} As $x\in\R,$ $\dfrac{1}{x}\in\R$, then by the Archimedean Principle (Theorem \ref{AP}): \[\exists n\in\Z\st n>\dfrac{1}{x}.\]\par\hspace{10mm} Hence, $nx>1$ or $x>\dfrac{1}{n}.$\par\hspace{10mm}As $x\leq\dfrac{1}{x},$ $n>\dfrac{1}{x}$, and $x>\dfrac{1}{n}$, we have \[\frac{1}{n}<x<n.\]\par$\boxed{\text{Case }2}$ If $x>1$, then $0<\dfrac{1}{x}<1.$\par\hspace{10mm} Hence, $\dfrac{1}{x}<1<x.$\par\hspace{10mm} As $x\in\R$, by the Archimedean Principle: \[\exists n\in\Z\st n>x>0\]\par\hspace{10mm} Hence, $\dfrac{1}{n}<\dfrac{1}{x}$\par\hspace{10mm} As $\dfrac{1}{x}<x$, $\dfrac{1}{n}<\dfrac{1}{x},$ and $n>x,$ we have \[\dfrac{1}{n}<x<n\]\par In all cases, we proven that $x\in\R,\ x>0\implies\exists n\in\Z, n>0\st\dfrac{1}{n}<x<n.$
\end{p}

\begin{framed}\begin{q}
	{Handout Chapter 1.4-2 More Contradictions and Equivelance}
	{There are no positive integer solutions to the equation $x^2-y^2=10$.}
\end{q}\end{framed}
\begin{p}
	Assume for the sake of contradiction that there are positive integer solutions to the equation $x^2-y^2=10.$\par Suppose $\exists x,y\in\Z$ and $x>0,\ y>0\st x^2-y^2=10.$\par Then, we have $x^2=10+y^2.$\par Since $x>0,\ x^2>0,$ we have $10+y^2>0.$\par Then, $y^2>-10.$\par\begin{center}$\divideontimes$ This contradicts with the fact that $y^2\geq0$ if $y\in\Z.$\end{center}\par So, our assumption is wrong. There must be no positive integer solutions to the equation $x^2-y^2=10.$
\end{p}

\begin{framed}\begin{q}
	{Handout Chapter 1.4-2 More Contradictions and Equivelance}
	{Show that if $a\in\Q$ and $b\in\Q'$, then $a+b\in\Q'$}
	\begin{rmk}
		The notation $\Q$ means the set for rational numbers, and $\Q'$ means the set for irrational numbers.
	\end{rmk}
\end{q}\end{framed}
\begin{p}
	Suppose $a\in\Q$ and $b\in\Q'$\par Assume for the sake of contradiction that $a+b\in\Q$.\par Then, $\exists m,n,p,q\in\Z$ such that $a=\dfrac{m}{n}$ and $a+b=\dfrac{p}{q}.$\par Then, \[b=\frac{p}{q}-a=\frac{p}{q}-\frac{m}{n}=\frac{pn-mq}{qn}\in\Q\]\par Since $pn-mq\in\Q$ and $qn\in\Z$, we have $b=\dfrac{pn-mq}{qn}\in\Q$.\par\begin{center}$\divideontimes$ This contradicts with the fact that $b\in\Q'$. \end{center}\par So, $a+b$ must be irrational. 
\end{p}

\begin{framed}\begin{q}
	{Handout Chapter 1.4-2 More Contradictions and Equivelance}
	{If $n\in\N$ and $2^n-1$ is prime, then $n$ is prime.}
\end{q}\end{framed}
\begin{p}
	We will prove the contrapositive: if $n$ is not prime, then $2^n-1$ is not prime.\par Suppose $n$ is not prime. Then, $\exists a,b\in\Z$ with $1<a,b<n\st n=ab.$\par Then, $2^n-1=2^{ab}=\left(2^a\right)^b-1.$\par Notice that for $x^w-1$, by polynomial long division, have \[x^w-1=(x-1)\left(x^{w-1}+x^{w-2}+\cdots+1\right),\]\par Substitute $x=2^a$ and $w=b$, we have \[2^n-1=\left(2^a-1\right)\left[\left(2^a\right)^{b-1}+\left(2^a\right)^{b-2}+\cdots+1\right].\]\par Since $\left(2^a-1\right)\in\Z$ and $\left[\left(2^a\right)^{b-1}+\left(2^a\right)^{b-2}+\cdots+1\right]\in\Z,$ we see that $2^n-1$ is not prime. 
\end{p}

\begin{framed}\begin{q}
	{Exam 1 Review 1-b-i}
	{Prove that $\qty[P\wedge\qty(P\implies Q)]\implies Q$.}
\end{q}\end{framed}
\hspace*{\fill}\newline
\hspace*{\fill}
\begin{p}
	\begin{center}\begin{tabular}{c|c|c|c|c}
		$P$&$Q$&$P\Rightarrow Q$&$P\wedge(P\Rightarrow Q)$&$\qty[P\wedge(P\Rightarrow Q)]\implies P$\\
		\hline
		T&T&T&T&T\\
		T&F&F&F&T\\
		F&T&T&F&T\\
		F&F&T&F&T
	\end{tabular}\end{center}
\end{p}

\begin{framed}\begin{q}
	{Exam 1 Review 1-b-ii}
	{Prove that $\qty[Q\wedge\qty(P\implies Q)]\implies P$.}
\end{q}\end{framed}
\begin{p}
	\begin{center}\begin{tabular}{c|c|c|c|c}
		$P$&$Q$&$P\Rightarrow Q$&$Q\wedge(P\Rightarrow Q)$&$\qty[Q\wedge(P\Rightarrow Q)]\implies Q$\\
		\hline
		T&T&T&T&T\\
		T&F&F&F&T\\
		F&T&T&T&T\\
		F&F&T&F&T
	\end{tabular}\end{center}
\end{p}

\begin{framed}\begin{q}
	{Exam 1 Review 2-a}
	{Given statements $P$ and $Q$, prove $\neg(P\vee Q)\equiv\neg P\wedge\neg Q$.}
\end{q}\end{framed}
\begin{p}
	\begin{center}\begin{tabular}{c|c|c|c|c|c|c}
		$P$&$Q$&$P\vee Q$&$\neg(P\vee Q)$&$\neg P$&$\neg Q$&$\neg P\wedge\neg Q$\\
		\hline
		T&T&T&F&F&F&F\\
		T&F&T&F&F&T&F\\
		F&T&T&F&T&F&F\\
		F&F&F&T&T&T&T
	\end{tabular}\end{center}
\end{p}

\begin{framed}\begin{q}
	{Exam 1 Review 2-b}
	{There is no smallest integer.}
\end{q}\end{framed}
\begin{p}
	Assume for the sake of contradiction that there exists a smallest integer $n$. Hence, $\forall x\in\Z,$ we have $x\geq n.$\par Notice that if $n>0,$ we have $0\in\Z$ and $0<n.$\par Hence, $n=0$ cannot be the smallest integer ($\divideontimes$)\par Therefore, $n$ most be smaller than $0$.\par Suppose $m=-n$.\par Since $n\in\Z$, $m=-n\in\Z\in\R$\par By the Archimedean Principle (Theorem \ref{AP}), $\exists k\in\Z\st k>m$.\par Hence, $k>-n.$\par Multiply $(-1)$ on both sides of the inequality: \[-k<n.\]\par As $k\in\Z$, $-k\in\Z$. Then $\exists -k\in\Z\st -k<n.$\par\begin{center}$\divideontimes$ This contradicts with our assumption that $n$ is the smallest integer. \end{center}\par Hence, our assumption must be wrong. There is no smallest integer. 
\end{p}

\begin{framed}\begin{q}
	{Exam 1 Review 2-c}
	{The number $\log_23$ is irrational.}
\end{q}\end{framed}
\begin{p}
	Assume for the sake of contradiction that $\log_2{3}$ is irrational.\par By definition, $\exists p,qin\Z,$ with $q\neq0\st\log_2{3}=\dfrac{p}{q}.$\par Observe that $\log_2{3}\neq0$. Then $p\neq0$ as well.\par By definition of logarithm, \[\begin{aligned}2^{p/q}&=3\\\qty(2^p)^{1/q}&=3\end{aligned}\]\par Raise two sides of the equation to the power of $q$: \[2^p=3^q\]\par As $p\neq0$ and $q\neq0,$ $2^p$ and $3^q$ are not $1\ \forall p,q\in\Z.$\par Hence, $2^p$ is even $\forall p\in\Z$ and $3^q$ is odd $\forall q\in\Z.$\par\begin{center}$\divideontimes$ This contradicts with the fact that an even number cannot equal to an odd number. \end{center}\par Hence, our assumption is wront. The number $\log_23$, then, must be irrational.
\end{p}

\begin{framed}\begin{q}
	{Exam 1 Review 2-d}
	{There is a rational number $a$ and an irrational number $b$ such that $a^b$ is rational.}
\end{q}\end{framed}
\begin{p}
	Observe that $1$ is a rational number and $\pi$ is an irrational number. \par Suppose $a=1$ and $b=\pi$, we have $a^b=a^\pi=1,$ which is irrational.	
\end{p}
\begin{p}
	Recall that we have proven in the previous proof, we have proven that $\log_23$ is an irrational number.\par Recall the definition of logarithm and exponents, we have \[2^{\log_23}=3\]\par Hence, we find a pair of $a$ and $b$ that  satisfies the requirement. 
\end{p}

\begin{framed}\begin{q}
	{Exam 1 Review 2-e}
	{For all integers $n$, the number $n+n^2+n^3+n^4$ is even.}
\end{q}\end{framed}
\begin{p}
	Suppose $n\in\Z$.\par$\boxed{\text{Case }1}$ If $n$ is even.\par\hspace{5mm}Suppose $n=2k\fs k\in\Z$.\par\hspace{5mm}Then,\[\begin{aligned}n+n^2+n^3+n^4&=(2k)+(2k)^2+(2k)^3+(2k)^4\\&=2k+4k^2+8k^3+16k^4\\&=2(k+2k^2+4k^3+8k^4)\end{aligned}\]\par\hspace{5mm}Since $(k+2k^2+4k^3+8k^4)\in\Z,$ we have $2(k+2k^2+4k^3+8k^4)$ is even.\par\hspace{5mm}Hence, $n+n^2+n^3+n^4$ is even when $n$ is even.\par$\boxed{\text{Case }2}$ If $n$ is odd.\par\hspace{5mm}Suppose $n=2k+1\fs k\in\Z.$\par\hspace{5mm}Then, \[\begin{aligned}n+n^2+n^3+n^4&=(2k+1)+(2k+1)^2+(2k+1)^3+(2k+1)^4\\&=2k+1+4k^2+4k+1+8k^3+12k^2+6k+1+16k^4+32k^3+24k^2+8k+1\\&=16k^4+40k^3+40k^2+20k+4\\&=2(8k^4+20k^3+20k^2+10k+2)\end{aligned}\]\par\hspace{5mm}Since $(8k^4+20k^3+20k^2+10k+2)\in\Z$, we have $2(8k^4+20k^3+20k^2+10k+2)$ is even.\par\hspace{5mm}Hence, $n+n^2+n^3+n^4$ is even when $n$ is odd.\par Since integers can either be even or odd, and we have proven $n+n^2+n^3+n^4$ is even in either case, $n+n^2+n^3+n^4$ is even for all integers. 
\end{p}

\begin{df}[Perfect Square]
A perfect square is an integer $n$ for which there exists an integer $m$ such that $n=m^2$. 
\end{df}

\begin{framed}\begin{q}
	{Exam 1 Review 2-f}
	{If $n$ is a positive integer such that $n$ is in the form $4k+2$ or $4k+3$, then $n$ is not a perfect square.}
\end{q}\end{framed}
\begin{p}
	We will prove the contrapositive of the statement: ``If $n$ is a perfect square, then $n$ is a positive integer of the form $4k$ or $4k+1\fs k\in\Z.$''\par Suppose $n$ to be a perfect square, then $\exists m\in\Z\st n=m^2.$\par $\boxed{\text{Case }1}$ Suppose $m$ is even, then $m=2t\fs t\in\Z.$\[n=m^2=(2t)^2=4t^2>0.\]\par \hspace*{10mm}Let $k=t^2$. Since $t^2\in\Z$, we have $k\in\Z.$\par\hspace*{10mm}Hence, $n$ is positive and is in the form of $4k$.\par $\boxed{\text{Case }2}$ Suppose $m$ is odd, then $m=2t+1\fs t\in\Z$.\[n=m^2=(2t+1)^2=4t^2+4t+1=4(t^2+t)+1>1\]\par\hspace*{10mm} Let $k=t^2+t$. Since $(t^2+t)\in\Z$, we have $k\in\Z.$\par\hspace*{10mm} Hence, $n$ is in the form of $4k+1$.\par Hence, we prove the contrapositive of the original statement to be true, which means our original statement is also true.	
\end{p}

\begin{framed}\begin{q}
	{Exam 1 Review 2-g}
	{For any integer $n$, $3\mid n$ if and only if $3\mid n^2$.}
\end{q}\end{framed}
\begin{p}
	Suppose $n\in\Z$.\par($\Rightarrow$) Suppose $3\mid n$. Then, $\exists k\in\Z\st n=3k.$\par\hspace{5mm}Then, $n^2=(3k)^2=9k^2=3(3k^2).$\par\hspace{5mm}Since $3k^2\in\Z$, by definition, $3\mid n^2.$\par($\Leftarrow$) WTS: $3\mid n^2\implies3\mid n$\par\hspace{5mm}We will prove the contrapositive: If $3\nmid n$, then $3\nmid n^2$\par\hspace{5mm}Suppose $3\nmid n.$\par\hspace{5mm}$\boxed{\text{Case }1}$ Suppose $n=3m+1\fs m\in\Z$.\par\hspace{10mm}Then, $n^2=(3m+1)^2=9m^2+6m+1$\par\hspace{10mm}Since $9m^2+6m+1$ cannot be written in the form of $3k\fs k\in\Z,$ by definition, $3\nmid n^2.$\par\hspace{5mm}$\boxed{\text{Case }2}$ Suppose $n=3m+2\fs m\in\Z$\par\hspace{10mm}Then, $n^2=(3m+2)^2=9m^2+12m+4$\par\hspace{10mm}Since $9m^2+12m+4$ cannot be written in the form of $3k$ for some $k\in\Z$, by definition, $3\nmid n^2.$\par\hspace{5mm} Hence, we proved the contrapositive, and thus the original statement is true.\par Therefore, $n\mid n\iff 3\mid n^2.$
\end{p}

\begin{framed}\begin{q}
	{Exam 1 Review 2-h}
	{There exists an integer $n$ such that $12\mid n^2$ but $12\nmid n$.}
\end{q}\end{framed}
\begin{p}
	Prove.	
\end{p}

\begin{framed}\begin{q}
	{Exam 1 Review 2-i}
	{For every integer $a$, the numbers $a$ and $(a+1)(a-1)$ have opposite parity.}
\end{q}\end{framed}
\begin{p}
	Prove.	
\end{p}

\begin{framed}\begin{q}
	{Exam 1 Review 2-j}
	{Suppose $x\in\R$. If $x^2$ is irrational, then $x$ is irrational.}
\end{q}\end{framed}
\begin{p}
	Prove.	
\end{p}

\begin{framed}\begin{q}
	{Exam 1 Review 2-k}
	{For any integers $a$ and $b$, if $ab$ is even, then $a$ is even or $b$ is even.}
\end{q}\end{framed}
\begin{p}
	Prove.	
\end{p}

\begin{framed}\begin{q}
	{Exam 1 Review 2-l}
	{For $n\in\N$, $n$, $n+2$, and $n+4$ are all prime if and only if $n=3$.}
\end{q}\end{framed}
\begin{p}
	Prove.	
\end{p}

\begin{framed}\begin{q}
	{Exam 1 Review 3-a}
	{Prove or disprove: Every real number is less than or equal to its square.}
\end{q}\end{framed}
\begin{dis}
	Prove.	
\end{dis}

\begin{framed}\begin{q}
	{Exam 1 Review 3-b}
	{Prove or disprove: The sum of two integers is never equal to their product.}
\end{q}\end{framed}
\begin{dis}
	Prove.	
\end{dis}

\begin{framed}\begin{q}
	{Exam 1 Review 3-c}
	{Prove or disprove: There exists a non-zero integer whose cube equals its negative.}
\end{q}\end{framed}
\begin{dis}
	Prove.	
\end{dis}

\begin{framed}\begin{q}
	{Exam 1 Review 3-d}
	{Prove or disprove: Fall all $x\in\R$, $x\leq x^2$ or $0\leq x<1$.}
\end{q}\end{framed}
\begin{p}
	Prove.	
\end{p}

\begin{framed}\begin{q}
	{Chapter 1.4 \# 20-a}
	{Let $n$ be an integer. Prove that $n$ is even if and only if $n^3$ is even.}
\end{q}\end{framed}
\begin{p}
	Prove.	
\end{p}

\begin{framed}\begin{q}
	{Chapter 1.4 \# 20-b}
	{Let $n$ be an integer. Prove that $n$ is odd if and only if $n^3$ is odd.}
\end{q}\end{framed}
\begin{p}
	Prove.	
\end{p}

\begin{framed}\begin{q}
	{Chapter 1.4 \# 21}
	{Prove that $\sqrt[3]{2}$ is irrational.}
\end{q}\end{framed}
\begin{p}
	Prove.	
\end{p}



\newpage
\section{Sets}
\begin{framed}\begin{q}
	{Handout Chapter 2.1 - Sets and Subsets}
	{Prove that $\{12a+4b\mid a,b\in\Z\}=\{4c\mid c\in\Z\}$.}
\end{q}\end{framed}
\begin{p}
	($\subseteq$) Suppose $x\in\qty{12a+4b\mid a,b\in\Z}$\par\hspace{5mm}Then, $x=12a+4b\fs a,b\in\Z.$\[x=12a+4b=4(3a+b)\]\par\hspace{5mm}As $3a+b\in\Z$, we have $x\in\qty{4c\mid c\in\Z}$\par\hspace{5mm}By definition, $\qty{12a+4b\mid a,b\in\Z}\subseteq\qty{4c\mid c\in\Z}$\par
	($\supseteq$) Suppose $x\in\qty{4c\mid c\in\Z}$\par\hspace{5mm}Then, $x=4c\fs c\in\Z.$\par\hspace{5mm}Suppose $c=3a+b\fs a,b\in\Z.$\par\hspace{5mm}Then, $x=4c=4(3a+b)=12a+4b.$\par\hspace{5mm}By definition, $\qty{4c\mid c\in\Z}\subseteq\qty{12a+4b\mid a,b\in\Z}$\par Hence, we have proven $\{12a+4b\mid a,b\in\Z\}=\{4c\mid c\in\Z\}.$
\end{p}

\begin{framed}\begin{q}
	{Exam 1 Review 2-m}
	{If $A=\{x\mid x=n^4-1,\ n\in\Z\}$ and $B=\{x\mid x=m^2-1,\ m\in\Z\}$, then $A\subseteq B$.}
\end{q}\end{framed}
\begin{p}
	Prove.	
\end{p}

\begin{framed}\begin{q}
	{Exam 1 Review 2-n}
	{If $A$, $B$, and $C$ are sets, then $A\cap(B\cup C)=(A\cap B)\cup(A\cap C)$.}
\end{q}\end{framed}
\begin{p}
	Prove.	
\end{p}

\begin{framed}\begin{q}
	{Exam 1 Review 2-o}
	{For subsets $A$ and $B$ of a universal set $U$, $\overline{A\cup B}=\overline{A}\cap\overline{B}$.}
\end{q}\end{framed}
\begin{p}
	Prove.	
\end{p}

\begin{framed}\begin{q}
	{Exam 1 Review 2-p}
	{Suppose that $A$, $B$, and $C$ are subsets of a universal set $U$. Let $P$ and $Q$ be the following statements:}\par\hspace{5mm}\texttt{$P$: $A\subseteq B$ or $A\subseteq C$; and}\par\hspace{5mm}\texttt{$Q$: $A\subseteq B\cap C$.}\par\noindent\texttt{Write the statement $P\Rightarrow Q$, its converse and contrapositive. Prove the truth statements among these or give counterexamples.}
\end{q}\end{framed}
\begin{p}
	Prove.	
\end{p}

\begin{framed}\begin{q}
	{Handout Chapter 2.2-Combining Sets}
	{Let $A=\{6a+4\mid x\in\Z\}$ and $B=\{18b-a\mid b\in\Z\}$. Prove or disprove: $A\subseteq B$.}
\end{q}\end{framed}
\begin{dis}
	Prove.	
\end{dis}

\begin{framed}\begin{q}
	{Handout Chapter 2.2-Combining Sets}
	{Let $A=\{6a+4\mid x\in\Z\}$ and $B=\{18b-a\mid b\in\Z\}$. Prove or disprove: $B\subseteq A$.}
\end{q}\end{framed}
\begin{p}
	Prove.	
\end{p}

\begin{framed}\begin{q}
	{Handout Chapter 2.2-Combining Sets}
	{If $A$ and $B$ are sets, then $\pwer(A)-\pwer(B)=\pwer(A-B)$.}
\end{q}\end{framed}
\begin{p}
	Prove.	
\end{p}

\begin{framed}\begin{q}
	{Handout Chapter 2.2-Combining Sets}
	{If $A$, $B$, and $C$ are sets, and $A\cross B=B\cross C$, then $A=B$.}
\end{q}\end{framed}
\begin{p}
	Prove.	
\end{p}

\begin{framed}\begin{q}
	{Chapter 2.1 \# 6}
	{Let $n\in\Z$ and let $A=n\Z$. Prove that if $x,y\in A,$ then $x+y\in Z$ and $xy\in A$.}
\end{q}\end{framed}
\begin{p}
	Prove.	
\end{p}



\label{LastPage}
\end{document}
\documentclass[12pt, a4paper]{article}

\usepackage[utf8]{inputenc}
\usepackage[framemethod=TikZ]{mdframed}
\usepackage[hidelinks]{hyperref}
\usepackage{mathtools, amssymb, amsmath, cleveref, fancyhdr, geometry, , graphicx, float, subfigure, arydshln, url, setspace, framed, pifont, physics, ntheorem}
%%% for coding %%%
\usepackage{listings}
\usepackage[ruled, vlined, linesnumbered]{algorithm2e}

\geometry{a4paper, left=2cm, right=2cm, bottom=2cm, top=2cm}

\pagestyle{fancy}
\fancyhead{}
\fancyhead[L]{\leftmark}
\fancyhead[R]{\rightmark}
\fancyfoot{}
\fancyfoot[C]{\thepage}
%\renewcommand{\headrulewidth}{0pt}
\renewcommand{\footrulewidth}{0pt}

\hypersetup{
	colorlinks = true,
	bookmarks = true,
	bookmarksnumbered = true,
	pdfborder = 001,
	linkcolor = grey
}

\definecolor{blue}{rgb}{0,0.45,1.14}
\definecolor{red}{rgb}{0.77,0.12,0.23}
\definecolor{grey}{rgb}{0.49,0.38,0.29}
\definecolor{orange}{rgb}{2.07,0.69,0.32}
\definecolor{purple}{rgb}{0.58,0,0.82}
\definecolor{mygreen}{rgb}{0,0.6,0}


%%% for coding %%%
\lstset{basicstyle = \ttfamily\small,commentstyle = \color{mygreen}\textit, deletekeywords = {...}, escapeinside = {\%*}{*)}, frame = single, framesep = 0.5em, keywordstyle = \bfseries\color{blue}, morekeywords = {*}, emph = {self}, emphstyle=\bfseries\color{red}, numbers = left, numbersep = 1.5em, numberstyle = \ttfamily\small\color{grey},  rulecolor = \color{black}, showstringspaces = false, stringstyle = \ttfamily\color{purple}, tabsize = 4, columns = flexible}

\theorembodyfont{\upshape}
\theoremseparator{.}
\theoremstyle{definition}
\newtheorem{cor}{Corollary}[section]
\newtheorem{lem}{Lemma}[section]
\newtheorem{ax}{Axiom}[section]
\newtheorem{prop}{Proposition}[subsection]
\newtheorem{conj}{Conjecture}[subsection]
\newtheorem{thm}{Theorem}[subsection]
\newtheorem{df}{Definition}[subsection]
\newtheorem{eg}{Example}[subsection]
\newenvironment*{ans}{\par\indent\textbf{\textit{Answer. }}\par}{\par\hfill{$\square$}\par}
\newtheorem*{rmk}{\indent Remark}
\newenvironment*{prf}{\par\indent\textbf{\textit{Proof. }}\par}{\par\hfill$\blacksquare$\par}
\newtheorem*{ext}{\indent Extension}

\linespread{1.5}

\title{\textbf{This is a title}}
\author{Jiuru Lyu}
\date{\today}

\def\Z{{\mathbb{Z}}}
\def\R{{\mathbb{R}}}
\def\C{{\mathbb{C}}}
\def\Q{{\mathbb{Q}}}
\def\E{{\mathbb{E}}}
\def\d{{\mathrm{d}}}
\def\i{{\mathrm{i}}}
\def\Arg{{\mathrm{Arg}}}
\def\cis{\mathrm{cis}}
\def\epsilon{\varepsilon}
\def\emptyset{\varemptyset}

\begin{document}
\maketitle

\tableofcontents

\section{This is the section name}
\subsection{This is the subsection name}

\begin{thm}[Name of thm]
	This is a theorem
\end{thm}

\begin{df}
	This is a definition
\end{df}

\begin{eg}
	This is an example
\end{eg}
\begin{ans}
	
\end{ans}


\begin{rmk}
	This is a remark
\end{rmk}

\begin{prf}
	This is a proof
\end{prf}

\begin{cor}
	This is a corollary
\end{cor}

\begin{lem}
	This is a lemma
\end{lem}

\begin{prop}
	This is a proposition
\end{prop}

\begin{conj}
	This is a conjecture
\end{conj}

\begin{ax}
	
\end{ax}

\begin{lstlisting}[language = Matlab, title = {Answer.m}]
% Plot function f(x) = 2*x^3 - x - 2
ezplot('2*x^3-x-2', [0, 2])
hold on
plot([0, 2], [0, 0], 'r')
\end{lstlisting}

\begin{lstlisting}[language = Python]
def __self__(self):
for i in range(10):
	print(f"This number is {i}.")
\end{lstlisting}

\begin{lstlisting}[language = java]
public static void main(String[] args) {
	System.out.println("Hello World!"); // comment
}
\end{lstlisting}

\begin{algorithm}
\caption{Bisection Algorithm}
\SetKwData{In}{\rmfamily\textbf{in}}\SetKwData{To}{\rmfamily\textbf{to}}\SetKwData{And}{\rmfamily{\textbf{and}}}\SetKwData{Or}{\rmfamily{\textbf{or}}}\SetKwData{Stop}{\rmfamily{\textbf{stop}}}\SetKwData{Break}{\rmfamily{\textbf{break}}}
\SetKwComment{Comment}{/* }{ */}
\DontPrintSemicolon
\KwIn{$a,b,M,\delta,\epsilon$\;$u\leftarrow f(a)$\;$b\leftarrow f(b)$\;$e\leftarrow b-a$}
\KwOut{output}
\BlankLine
\Begin{
	\If{$\text{sign}(u)=\text{sign}(v)$}{\Stop}
	\For{k=1 \To M}{
		$e\leftarrow e/2$\;
		$c\leftarrow a+e$\;
		$w\leftarrow f(c)$\;
		\Return $k,c,w,e$\;
		\If{$|e|<\delta$\Or$|w|<\epsilon$}{\Stop}
		\eIf{$\text{sign}(u)\neq=\text{sign}(v)$}{
			$b\leftarrow c$\;
			$v\leftarrow w$\;
		} {
			$a\leftarrow c$\;
			$u\leftarrow w$\;
		}
	}
}
\end{algorithm}
\end{document}